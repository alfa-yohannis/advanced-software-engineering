\documentclass[11pt,b5paper]{book}

\usepackage{parskip}
\setlength{\parskip}{0.5em}
\setlength{\parindent}{1.5em}

\usepackage{longtable}
\usepackage{xcolor}
\usepackage{fontspec}
\setmainfont[
Path = fonts/, 
UprightFont = TitilliumWeb-Regular.ttf,
BoldFont = TitilliumWeb-Bold.ttf,
ItalicFont = TitilliumWeb-Italic.ttf,
BoldItalicFont = TitilliumWeb-BoldItalic.ttf
]{TitilliumWeb}

%\usepackage[utf8]{inputenc}
%\usepackage{listingsutf8}
%\lstset{inputencoding=utf8}

\usepackage{pgfplots}
\usepgfplotslibrary{polar}
\pgfplotsset{compat=1.18}

\usepackage{tikz}
\usetikzlibrary{shapes.geometric, 
	positioning, shapes, shapes.multipart, backgrounds, arrows.meta, calc, fit, trees}

\usepackage{amsmath}

\usepackage{tabularx}
\usepackage[b5paper,inner=2.5cm,
  outer=2.0cm,
  top=2.5cm,
  bottom=2.5cm]{geometry}
\usepackage[colorlinks=true, linkcolor=black, citecolor=blue, urlcolor=blue]{hyperref}
\usepackage{graphicx}
\usepackage{xcolor}
\graphicspath{{images/}}
\usepackage{svg}

\usepackage{subcaption}

\hyphenation{}

\usepackage{listings}


% Define Python language style for listings
\lstdefinestyle{PythonStyle}{
    language=Python,
    basicstyle=\ttfamily\footnotesize,
    keywordstyle=\color{blue}\bfseries,
    commentstyle=\color{gray}\itshape,
    stringstyle=\color{red},
    showstringspaces=false,
    breaklines=true,
    frame=lines,
    numbers=left,
    numberstyle=\tiny\color{gray},
    backgroundcolor=\color{lightgray!10},
    tabsize=2,
    captionpos=b
}

\lstdefinestyle{SqlStyle}{
	language=SQL,
	basicstyle=\ttfamily\footnotesize,
	morekeywords={REAL, TEXT, REFERENCES},
	keywordstyle=\color{blue},
	commentstyle=\color{gray},
	stringstyle=\color{red},
	breaklines=true,
	showstringspaces=false,
	tabsize=2,
	captionpos=b,
	numbers=left,
	numberstyle=\tiny\color{gray},
	frame=lines,
	backgroundcolor=\color{lightgray!10},
	comment=[l]{//},
	morecomment=[s]{/*}{*/},
	commentstyle=\color{gray}\ttfamily,
	string=[s]{'}{'},
	morestring=[s]{"}{"},
	%	stringstyle=\color{teal}\ttfamily,
	%	showstringspaces=false
}


\lstdefinelanguage{json}{
	basicstyle=\ttfamily\footnotesize,
	morekeywords={REAL, TEXT, REFERENCES},
	keywordstyle=\color{blue},
	commentstyle=\color{gray},
	stringstyle=\color{red},
	breaklines=true,
	showstringspaces=false,
	tabsize=2,
	captionpos=b,
	numbers=left,
	numberstyle=\tiny\color{gray},
	frame=lines,
	backgroundcolor=\color{lightgray!10},
	comment=[l]{//},
	morecomment=[s]{/*}{*/},
	commentstyle=\color{gray}\ttfamily,
	string=[s]{'}{'},
	morestring=[s]{"}{"},
	%	stringstyle=\color{teal}\ttfamily,
	%	showstringspaces=false
}

\lstdefinelanguage{bash} {
	keywords={},
	basicstyle=\ttfamily\small,
	keywordstyle=\color{blue}\bfseries,
	ndkeywords={iex},
	ndkeywordstyle=\color{purple}\bfseries,
	sensitive=true,
	commentstyle=\color{gray},
	stringstyle=\color{red},
	numbers=left,
	numberstyle=\tiny\color{gray},
	breaklines=true,
	frame=lines,
	backgroundcolor=\color{lightgray!10},
	tabsize=2,
	comment=[l]{\#},
	morecomment=[s]{/*}{*/},
	commentstyle=\color{gray}\ttfamily,
	stringstyle=\color{purple}\ttfamily,
	showstringspaces=false
}

% Define Java language style for listings
\lstdefinestyle{JavaScript}{
	language=Java,
	basicstyle=\ttfamily\footnotesize,
	keywordstyle=\color{blue},
	commentstyle=\color{gray},
	stringstyle=\color{red},
	breaklines=true,
	showstringspaces=false,
	tabsize=4,
	captionpos=b,
	numbers=left,
	numberstyle=\tiny\color{gray},
	frame=lines,
	backgroundcolor=\color{lightgray!10},
	comment=[l]{//},
	morecomment=[s]{/*}{*/},
	commentstyle=\color{gray}\ttfamily,
	string=[s]{'}{'},
	morestring=[s]{"}{"},
	%	stringstyle=\color{teal}\ttfamily,
	%	showstringspaces=false
}

\usepackage{plantuml}


\newcommand{\authors}[1]{{\chapterauthor{#1}\addtocontents{toc}{#1\par}}}
\newcommand{\HUGE}{\fontsize{40}{48}\selectfont}

\renewcommand{\contentsname}{Daftar Isi}
\renewcommand{\chaptername}{Bab}
\renewcommand{\figurename}{Gambar}

\definecolor{praditagreen}{RGB}{37, 113, 71}

\makeatletter
\newcommand{\chapterauthor}[1]{%
	{\parindent0pt\vspace*{-25pt}%
		\linespread{1.1}\large\scshape#1%
		\par\nobreak\vspace*{35pt}}
	\@afterheading%
}
\makeatother

\begin{document}


	\begin{titlepage}
	\begin{tikzpicture}[remember picture, overlay]
		% Gambar latar belakang atau logo
		\node[anchor=north west, inner sep=0pt] at ([xshift=.18\textwidth, yshift=-.1\textwidth] current page.north west){
			\includegraphics[width=.3\paperwidth]{../figures/logo_universitas.png} 
		};
	
		\node[anchor=north west, inner sep=0pt, opacity=.12] at (current page.north west){
			\includegraphics[width=\paperwidth]{../figures/building.png}
		};
		
		% Judul Tengah
		\node[anchor=center] at ([yshift=.1\textheight]current page.center) {
			\begin{minipage}{\textwidth}
				
				{\LARGE \textbf{Course Module -- IT30213}}\\[0.03\textheight] % Book Type
				{\HUGE \textcolor{praditagreen}{
				\textbf{Advanced\vspace{10pt}\\
						Software Engineering
				}}}\\[.05\textheight] % Title
%				{\Huge \textcolor{praditagreen}{\textbf{Penggunaan Large-language Model}}}\\[.08\textheight] % Subtitle
				{\LARGE \textbf{Alfa Yohannis}}\\[0.05\textheight] % Authors
			\end{minipage}
		};
		
		% Teks bawah dengan latar oranye penuh dan teks putih
		\node[anchor=south west] at (
		current page.south west) {
			\begin{tikzpicture}[remember picture, overlay]
				\node[anchor=south west, fill=praditagreen, text=white, minimum width=\paperwidth, minimum height=.14\textheight, align=center, font=\Huge\bfseries] at (-.15,-.15) {
					Department of Information Technology\\
					Pradita University
				};
			\end{tikzpicture}
		};
	\end{tikzpicture}
\end{titlepage}
	
	% Contents Page
	\tableofcontents
	
	\chapter{Pendahuluan}

\section{Isi Mata Kuliah}

\subsection*{Sesi 1 --- Sistem Berjalan}
\textbf{Deskripsi:} Sesi ini memperkenalkan mahasiswa pada sebuah sistem perangkat lunak yang benar-benar berjalan, dengan tujuan agar seluruh abstraksi pada sesi berikutnya berangkat dari perilaku runtime yang konkret, sebelum konsep arsitektur atau pemodelan apa pun diperkenalkan.  
\textbf{Alat:} Docker.  
\textbf{Mahasiswa Non-IT:} Mahasiswa menjalankan aplikasi berbasis container, memodifikasi parameter lingkungan, serta mengamati perubahan perilaku sistem, sambil menjelaskan kegagalan dan efek konfigurasi menggunakan bahasa alami.  
\textbf{Mahasiswa IT:} Mahasiswa membangun dan menjalankan container, mengatur port dan environment variable, serta mendiagnosis kegagalan runtime langsung pada level container.

\subsection*{Sesi 2 --- Observabilitas}
\textbf{Deskripsi:} Mahasiswa belajar mengamati perilaku sistem sebelum menalar struktur atau desain, dengan menjadikan data runtime sebagai sumber utama pemahaman sistem.  
\textbf{Alat:} Prometheus, Grafana, dan konsep OpenTelemetry.  
\textbf{Mahasiswa Non-IT:} Mahasiswa menafsirkan dashboard, metrik, dan log, serta mengaitkan nilai yang diamati dengan perilaku yang terlihat oleh pengguna dan kegagalan sistem.  
\textbf{Mahasiswa IT:} Mahasiswa mengonfigurasi telemetri dasar, memvisualisasikan log dan trace, serta mengaitkan data observabilitas dengan peristiwa runtime.

\subsection*{Sesi 3 --- Kontrol Runtime, Keamanan, dan Kebijakan}
\textbf{Deskripsi:} Keamanan dan kepatuhan diperkenalkan sebagai mekanisme yang secara aktif membatasi perilaku runtime sistem, bukan sekadar sebagai aktivitas audit eksternal.  
\textbf{Alat:} Docker, pengantar konsep Kubernetes, dan konsep Terraform.  
\textbf{Mahasiswa Non-IT:} Mahasiswa mengidentifikasi aturan keamanan, batasan, serta pelanggaran kebijakan, dan menjelaskan dampaknya terhadap risiko dan kepatuhan sistem.  
\textbf{Mahasiswa IT:} Mahasiswa menerapkan kontrol keamanan dasar pada container dan deployment, serta mengonfigurasi batasan runtime pada level lingkungan.

\subsection*{Sesi 4 --- CI/CD dan Ledakan Skrip}
\textbf{Deskripsi:} Mahasiswa mengalami langsung keterbatasan otomasi berbasis skrip melalui pipeline continuous integration dan deployment, dengan menyoroti duplikasi dan kerapuhan solusi skrip.  
\textbf{Alat:} CI/CD berbasis YAML dan Bash.  
\textbf{Mahasiswa Non-IT:} Mahasiswa menelusuri langkah eksekusi pipeline dan mendokumentasikan masalah pemeliharaan otomasi.  
\textbf{Mahasiswa IT:} Mahasiswa memodifikasi pipeline CI/CD lintas lingkungan dan mengamati bagaimana perubahan kecil pada skrip dapat memicu kegagalan runtime.

\subsection*{Sesi 5 --- Model Runtime dan Perilaku Eksekutabel}
\textbf{Deskripsi:} Artefak runtime dieksplisitkan sebagai model dan langsung diperlakukan sebagai struktur keputusan yang dapat dieksekusi, menjembatani transisi dari konfigurasi implisit ke eksekusi yang terkendali.  
\textbf{Alat:} Pemodelan konseptual, representasi YAML atau JSON, serta interpreter Python sederhana.  
\textbf{Mahasiswa Non-IT:} Mahasiswa mengidentifikasi struktur runtime sebagai model dan menjelaskan bagaimana eksekusi mengikuti keputusan dalam model secara bertahap.  
\textbf{Mahasiswa IT:} Mahasiswa mengeksternalisasi keputusan runtime ke dalam model eksplisit dan mengimplementasikan interpreter dasar untuk mengeksekusi model tersebut.

\subsection*{Sesi 6 --- Dasar-Dasar Metamodeling}
\textbf{Deskripsi:} Mahasiswa mendefinisikan struktur model yang valid dengan memperkenalkan metamodel sebagai sistem tipe yang memformalkan perhatian runtime dan batas eksekusi.  
\textbf{Alat:} EMF dan Ecore.  
\textbf{Mahasiswa Non-IT:} Mahasiswa menafsirkan diagram metamodel dan menalar apakah contoh model bersifat valid atau tidak valid.  
\textbf{Mahasiswa IT:} Mahasiswa merancang metamodel Ecore yang secara formal menangkap konsep runtime, pipeline, dan kebijakan.

\subsection*{Sesi 7 --- Rekayasa DSL Tekstual}
\textbf{Deskripsi:} Sebuah domain-specific language (DSL) tekstual direkayasa untuk menggantikan praktik konfigurasi dan skrip yang terfragmentasi dengan sintaks konkret yang koheren.  
\textbf{Alat:} Xtext.  
\textbf{Mahasiswa Non-IT:} Mahasiswa membaca dan menulis contoh DSL serta mengevaluasi sejauh mana bahasa tersebut mengekspresikan maksud domain dengan jelas.  
\textbf{Mahasiswa IT:} Mahasiswa merancang grammar, mengimplementasikan parsing, serta mendefinisikan aturan scoping dan linking.

\subsection*{Sesi 8 --- Rekayasa Bahasa Visual untuk Model Runtime}
\textbf{Deskripsi:} Bahasa visual diperkenalkan sebagai sintaks konkret yang dapat dieksekusi di atas sintaks abstrak yang sama, dengan penekanan pada pengeditan dan validasi model, bukan sekadar dokumentasi.  
\textbf{Alat:} Sirius atau Papyrus.  
\textbf{Mahasiswa Non-IT:} Mahasiswa mengeksplorasi, mengedit, dan memvalidasi model runtime menggunakan editor visual untuk menalar struktur dan batasan.  
\textbf{Mahasiswa IT:} Mahasiswa mendefinisikan viewpoint visual, mapping, dan constraint yang mengaitkan elemen metamodel dengan representasi visual yang dapat dieksekusi.

\subsection*{Sesi 9 --- Rekayasa Bahasa Proyeksional}
\textbf{Deskripsi:} Rekayasa bahasa proyeksional diperkenalkan untuk menunjukkan komposisi dan evolusi bahasa yang sadar struktur tanpa keterbatasan parsing.  
\textbf{Alat:} JetBrains MPS.  
\textbf{Mahasiswa Non-IT:} Mahasiswa membandingkan bahasa tekstual, visual, dan proyeksional, serta memperluas bahasa menggunakan editor terstruktur.  
\textbf{Mahasiswa IT:} Mahasiswa mengimplementasikan konstruksi bahasa proyeksional dan mengomposisikan beberapa DSL menjadi satu bahasa terpadu.

\subsection*{Sesi 10 --- Desain Semantik Model Eksekutabel}
\textbf{Deskripsi:} Model diberikan makna operasional yang presisi dengan mendefinisikan bagaimana elemen model berperilaku saat dieksekusi, serta membedakan antara interpretasi dan generasi.  
\textbf{Alat:} Java, Kotlin, atau Python.  
\textbf{Mahasiswa Non-IT:} Mahasiswa menjelaskan semantik eksekusi menggunakan skenario runtime konkret dan penalaran terstruktur.  
\textbf{Mahasiswa IT:} Mahasiswa mengimplementasikan interpreter atau mesin semantik yang mengeksekusi model secara langsung.

\subsection*{Sesi 11 --- Transformasi Model ke Model dan Refinement}
\textbf{Deskripsi:} Model ditransformasikan dan diperkaya secara sistematis untuk mendukung pengelolaan abstraksi, variabilitas, dan spesialisasi lingkungan, sambil tetap menjaga semantik model.  
\textbf{Alat:} Epsilon ETL atau alat transformasi model-ke-model yang setara.  
\textbf{Mahasiswa Non-IT:} Mahasiswa menalar bagaimana model berevolusi lintas tingkat abstraksi menggunakan contoh-contoh konkret.  
\textbf{Mahasiswa IT:} Mahasiswa mengimplementasikan transformasi model-ke-model yang memperkaya model abstrak menjadi varian yang siap dieksekusi.

\subsection*{Sesi 12 --- Constraint dan Validasi Model}
\textbf{Deskripsi:} Constraint terkait kebenaran, keselamatan, dan keamanan diterapkan pada model sejak tahap desain untuk mencegah konfigurasi runtime yang tidak valid atau berbahaya.  
\textbf{Alat:} OCL terintegrasi dengan model berbasis EMF.  
\textbf{Mahasiswa Non-IT:} Mahasiswa merumuskan aturan kebenaran dan kepatuhan serta mengidentifikasi skenario model yang tidak valid.  
\textbf{Mahasiswa IT:} Mahasiswa mengimplementasikan constraint OCL dan mengintegrasikan validasi ke dalam alur kerja pemodelan.

\subsection*{Sesi 13 --- Generasi Model ke Teks dan Eksekusi DevSecOps}
\textbf{Deskripsi:} Model yang telah tervalidasi diterjemahkan menjadi artefak yang dapat dieksekusi, menutup siklus antara pemodelan dan eksekusi DevSecOps otomatis.  
\textbf{Alat:} Epsilon EGL, generator Xtext atau MPS, dengan target pipeline CI/CD, Kubernetes, dan Terraform.  
\textbf{Mahasiswa Non-IT:} Mahasiswa menelusuri bagaimana model tervalidasi diubah menjadi artefak runtime dan menalar keterlacakan serta kebenarannya.  
\textbf{Mahasiswa IT:} Mahasiswa mengimplementasikan generator model-ke-teks dan mengintegrasikan artefak hasil generasi ke dalam pipeline eksekusi otomatis.

	\chapter{Struktur Proyek dan Perkakas Flutter}

\section{Pendahuluan}

Flutter merupakan kerangka kerja (framework) pengembangan aplikasi lintas platform yang memungkinkan pengembang membangun aplikasi mobile, web, dan desktop dari satu basis kode yang sama. Untuk dapat memanfaatkan Flutter secara efektif, seorang pengembang tidak cukup hanya memahami pembuatan antarmuka pengguna, tetapi juga harus menguasai struktur proyek, perkakas pendukung, serta alur kerja pengembangan yang disediakan oleh ekosistem Flutter.

Pertemuan pertama ini berfokus pada pemahaman dasar mengenai bagaimana sebuah proyek Flutter disusun dan bagaimana berbagai perkakas digunakan dalam proses pengembangan aplikasi. Pemahaman terhadap struktur proyek menjadi fondasi penting karena akan memengaruhi keterbacaan kode, kemudahan pemeliharaan, serta skalabilitas aplikasi di masa depan. Kesalahan pada tahap awal, seperti pengelolaan dependensi yang kurang tepat atau struktur proyek yang tidak konsisten, dapat menimbulkan masalah serius pada tahap pengembangan lanjutan.

Selain struktur proyek, Flutter juga menyediakan berbagai perkakas yang mendukung siklus hidup pengembangan aplikasi, mulai dari proses instalasi, pengujian, hingga build untuk distribusi. Melalui Flutter Command Line Interface (CLI), pengembang dapat menjalankan aplikasi, mengelola dependensi, melakukan build, serta mendiagnosis permasalahan lingkungan pengembangan secara efisien.

Pada pertemuan ini, mahasiswa diharapkan memperoleh gambaran menyeluruh mengenai ekosistem pengembangan Flutter, mencakup instalasi dan konfigurasi lingkungan pengembangan, pemahaman struktur dasar proyek, penggunaan berkas konfigurasi \texttt{pubspec.yaml}, serta konsep build dan flavor aplikasi. Materi ini menjadi dasar bagi pertemuan-pertemuan selanjutnya yang akan membahas pengembangan antarmuka, manajemen state, integrasi data, hingga optimasi performa aplikasi Flutter.


\section{Instalasi dan Lingkungan Pengembangan Flutter}

\subsection{Persyaratan Sistem}

Sebelum melakukan instalasi Flutter, penting untuk memastikan bahwa sistem yang digunakan telah memenuhi persyaratan minimum agar proses pengembangan dapat berjalan dengan lancar. Kesesuaian spesifikasi perangkat keras dan perangkat lunak akan sangat memengaruhi stabilitas lingkungan pengembangan, waktu kompilasi, serta kenyamanan dalam menjalankan emulator atau perangkat fisik.

Secara umum, Flutter dapat dijalankan pada berbagai sistem operasi utama, yaitu Windows, macOS, dan Linux. Meskipun demikian, masing-masing platform memiliki ketentuan dan ketergantungan tambahan, terutama terkait dengan pengembangan aplikasi Android dan iOS. Untuk pengembangan iOS, Flutter hanya dapat dijalankan pada sistem operasi macOS karena ketergantungan terhadap Xcode dan perangkat pengembangan iOS.

Dari sisi perangkat keras, pengembangan Flutter disarankan dilakukan pada komputer dengan prosesor modern yang mendukung arsitektur 64-bit. Kapasitas memori (RAM) yang memadai sangat penting, terutama ketika menjalankan emulator Android atau simulator iOS secara bersamaan dengan lingkungan pengembangan. Ruang penyimpanan yang cukup juga diperlukan untuk mengakomodasi Flutter SDK, Android SDK, emulator, serta dependensi proyek.

Selain perangkat keras, sistem operasi dan perangkat lunak pendukung juga harus diperhatikan. Flutter memerlukan Git untuk proses instalasi dan pembaruan SDK. Java Development Kit (JDK) dibutuhkan untuk pengembangan Android, sementara Android Studio atau IDE lain yang mendukung Flutter dapat digunakan sebagai lingkungan pengembangan utama. Untuk memastikan kompatibilitas dan kemudahan instalasi, disarankan menggunakan versi stabil terbaru dari sistem operasi dan perangkat lunak pendukung.

Dengan memastikan seluruh persyaratan sistem terpenuhi sejak awal, pengembang dapat menghindari berbagai permasalahan teknis yang sering muncul pada tahap instalasi dan konfigurasi lingkungan pengembangan. Hal ini juga akan mempermudah proses pembelajaran dan pengembangan aplikasi Flutter pada tahap-tahap selanjutnya.

\subsection{Instalasi Flutter SDK}

Flutter SDK merupakan komponen utama yang menyediakan pustaka, perkakas, dan runtime yang dibutuhkan untuk mengembangkan aplikasi Flutter. Instalasi Flutter SDK dilakukan melalui terminal atau command prompt dan dapat diterapkan pada berbagai sistem operasi seperti Windows, macOS, dan Linux.

Langkah pertama adalah mengunduh Flutter SDK resmi dari repositori Flutter. Setelah proses unduhan selesai, SDK diekstrak ke dalam direktori tertentu yang mudah diakses, misalnya direktori \texttt{home}. Contoh proses ekstraksi Flutter SDK pada sistem berbasis Unix ditunjukkan pada perintah berikut.

\begin{lstlisting}[language=bash]
$ cd ~
$ tar xf flutter_linux_3.x.x-stable.tar.xz
\end{lstlisting}

Setelah proses ekstraksi selesai, direktori Flutter perlu ditambahkan ke dalam variabel lingkungan \texttt{PATH} agar perintah \texttt{flutter} dapat dijalankan dari direktori mana pun. Penambahan \texttt{PATH} dapat dilakukan dengan mengedit berkas konfigurasi shell, seperti \texttt{.bashrc} atau \texttt{.zshrc}.

\begin{lstlisting}[language=bash]
$ export PATH="$PATH:$HOME/flutter/bin"
\end{lstlisting}

Agar perubahan konfigurasi dapat langsung digunakan, berkas konfigurasi shell perlu dimuat ulang atau sesi terminal dibuka kembali.

\begin{lstlisting}[language=bash]
$ source ~/.bashrc
\end{lstlisting}

Setelah konfigurasi \texttt{PATH} selesai, langkah selanjutnya adalah memastikan Flutter SDK telah terinstal dengan benar menggunakan perintah diagnostik \texttt{flutter doctor}. Perintah ini akan memeriksa kelengkapan lingkungan pengembangan, termasuk Flutter SDK, Android SDK, serta perkakas pendukung lainnya.

\begin{lstlisting}[language=bash]
$ flutter doctor
\end{lstlisting}

Output dari perintah \texttt{flutter doctor} akan menampilkan status setiap komponen dalam bentuk checklist. Jika terdapat komponen yang belum terinstal atau belum dikonfigurasi dengan benar, Flutter akan memberikan rekomendasi langkah perbaikan yang dapat diikuti oleh pengembang.

Flutter menyediakan beberapa kanal rilis, seperti \texttt{stable}, \texttt{beta}, dan \texttt{dev}. Untuk keperluan pembelajaran dan pengembangan aplikasi produksi, disarankan menggunakan kanal \texttt{stable}. Pengelolaan kanal rilis Flutter dapat dilakukan melalui perintah berikut.

\begin{lstlisting}[language=bash]
$ flutter channel stable
$ flutter upgrade
\end{lstlisting}

Dengan selesainya instalasi Flutter SDK dan verifikasi lingkungan pengembangan, pengembang telah menyiapkan fondasi yang kuat untuk memulai pembuatan proyek Flutter, konfigurasi IDE, serta pengembangan fitur aplikasi pada tahap-tahap selanjutnya.

\subsection{Konfigurasi Android Studio dan VS Code}

Setelah Flutter SDK berhasil diinstal, langkah berikutnya adalah menyiapkan Integrated Development Environment (IDE) yang akan digunakan untuk proses pengembangan aplikasi. Flutter mendukung berbagai IDE, namun Android Studio dan Visual Studio Code (VS Code) merupakan dua pilihan yang paling umum digunakan karena dukungan plugin yang lengkap dan integrasi yang baik dengan Flutter CLI.

\subsubsection{Konfigurasi Android Studio}

Android Studio merupakan IDE resmi untuk pengembangan aplikasi Android dan menyediakan integrasi yang erat dengan Flutter. Untuk menggunakan Flutter pada Android Studio, langkah pertama adalah memastikan bahwa Android Studio telah terinstal pada sistem. Setelah instalasi, plugin Flutter dan Dart perlu ditambahkan melalui menu pengelolaan plugin.

Setelah plugin terpasang, Android Studio akan secara otomatis mendeteksi Flutter SDK yang telah diinstal. Jika Flutter SDK belum terdeteksi, pengembang dapat mengatur lokasi SDK secara manual melalui menu pengaturan Android Studio. Selain itu, Android Studio juga menyediakan fitur pengelolaan Android SDK dan emulator yang diperlukan untuk menjalankan dan menguji aplikasi Flutter.

Pembuatan emulator Android dilakukan melalui Android Virtual Device (AVD) Manager. Emulator ini memungkinkan pengembang menjalankan dan menguji aplikasi Flutter tanpa memerlukan perangkat fisik. Disarankan untuk menggunakan image sistem yang mendukung arsitektur dan versi Android terbaru yang stabil.

Untuk memastikan seluruh komponen Android telah terkonfigurasi dengan benar, pengembang dapat kembali menjalankan perintah diagnostik Flutter.

\begin{lstlisting}[language=bash]
$ flutter doctor
\end{lstlisting}

Jika terdapat komponen Android yang belum terinstal, Flutter akan menampilkan instruksi tambahan, seperti pemasangan Android SDK atau penerimaan lisensi Android.

\begin{lstlisting}[language=bash]
$ flutter doctor --android-licenses
\end{lstlisting}

\subsubsection{Konfigurasi Visual Studio Code}

Visual Studio Code (VS Code) merupakan editor kode ringan yang banyak digunakan untuk pengembangan Flutter karena cepat dan fleksibel. Untuk menggunakan Flutter di VS Code, pengembang perlu memasang ekstensi Flutter yang secara otomatis akan menyertakan dukungan untuk bahasa Dart.

Setelah ekstensi Flutter dan Dart terpasang, VS Code akan mendeteksi Flutter SDK dan mengaktifkan berbagai fitur pendukung, seperti code completion, debugging, hot reload, dan manajemen proyek Flutter. Pengembang juga dapat memilih emulator atau perangkat fisik langsung dari status bar VS Code untuk menjalankan aplikasi.

VS Code terintegrasi langsung dengan Flutter CLI, sehingga berbagai perintah Flutter dapat dijalankan melalui terminal bawaan. Untuk memastikan konfigurasi berjalan dengan baik, pengembang dapat menjalankan proyek Flutter contoh menggunakan perintah berikut.

\begin{lstlisting}[language=bash]
$ flutter create hello_flutter
$ cd hello_flutter
$ flutter run
\end{lstlisting}

Apabila aplikasi contoh berhasil dijalankan pada emulator atau perangkat fisik, maka konfigurasi IDE dan lingkungan pengembangan Flutter dapat dianggap telah selesai dengan baik.

Dengan selesainya konfigurasi Android Studio dan VS Code, pengembang memiliki lingkungan pengembangan yang siap digunakan untuk membangun, menjalankan, dan melakukan debugging aplikasi Flutter secara efektif. Lingkungan ini akan digunakan secara konsisten pada pertemuan-pertemuan berikutnya untuk mengembangkan berbagai fitur aplikasi Flutter.

\subsection{Verifikasi Instalasi dengan Flutter Doctor}

Setelah seluruh komponen utama Flutter SDK dan lingkungan pengembangan dikonfigurasi, langkah penting berikutnya adalah melakukan verifikasi instalasi. Flutter menyediakan sebuah alat diagnostik bernama \texttt{flutter doctor} yang berfungsi untuk memeriksa kelengkapan dan kesiapan lingkungan pengembangan sebelum aplikasi Flutter dijalankan atau dibangun.

Perintah \texttt{flutter doctor} dijalankan melalui terminal atau command prompt. Perintah ini akan melakukan serangkaian pemeriksaan terhadap Flutter SDK, Dart SDK, Android SDK, perangkat pengembangan, serta berbagai perkakas pendukung lainnya.

\begin{lstlisting}[language=bash]
$ flutter doctor
\end{lstlisting}

Hasil dari perintah \texttt{flutter doctor} ditampilkan dalam bentuk daftar komponen beserta statusnya. Setiap komponen ditandai dengan simbol tertentu yang menunjukkan kondisi instalasi, seperti tanda centang untuk komponen yang telah terpasang dengan benar, tanda peringatan untuk komponen yang memerlukan konfigurasi tambahan, atau tanda silang untuk komponen yang belum tersedia.

Apabila Flutter Doctor mendeteksi bahwa beberapa komponen Android belum dikonfigurasi sepenuhnya, pengembang biasanya diminta untuk menyetujui lisensi Android SDK. Proses ini dapat dilakukan langsung melalui perintah berikut.

\begin{lstlisting}[language=bash]
$ flutter doctor --android-licenses
\end{lstlisting}

Setelah lisensi disetujui, disarankan untuk menjalankan kembali perintah \texttt{flutter doctor} guna memastikan bahwa seluruh permasalahan telah terselesaikan.

\begin{lstlisting}[language=bash]
$ flutter doctor
\end{lstlisting}

Selain pemeriksaan dasar, Flutter Doctor juga dapat digunakan untuk memverifikasi ketersediaan perangkat pengembangan, baik berupa emulator maupun perangkat fisik. Informasi ini penting untuk memastikan bahwa aplikasi Flutter dapat dijalankan dan diuji secara langsung pada lingkungan yang sesuai.

Dengan memastikan bahwa seluruh komponen yang diperiksa oleh Flutter Doctor berada dalam kondisi siap, pengembang dapat melanjutkan ke tahap pembuatan dan pengembangan aplikasi Flutter tanpa hambatan teknis yang signifikan. Verifikasi ini juga sebaiknya dilakukan secara berkala, terutama setelah pembaruan Flutter SDK atau perubahan konfigurasi lingkungan pengembangan.

\section{Struktur Dasar Proyek Flutter}

\subsection{Struktur Direktori Proyek}

Setiap proyek Flutter memiliki struktur direktori standar yang secara otomatis dihasilkan ketika sebuah proyek baru dibuat. Struktur ini dirancang untuk memisahkan kode aplikasi, konfigurasi platform, serta aset pendukung, sehingga pengembangan dan pemeliharaan aplikasi dapat dilakukan secara terorganisasi dan konsisten.

Pemahaman terhadap struktur direktori proyek Flutter merupakan hal yang sangat penting, terutama bagi pengembang pemula. Dengan memahami fungsi masing-masing direktori dan berkas, pengembang dapat mengetahui di mana kode aplikasi seharusnya ditulis, bagaimana konfigurasi platform dilakukan, serta bagaimana aset seperti gambar dan font dikelola.

Struktur dasar sebuah proyek Flutter yang baru dibuat dapat dilihat melalui perintah berikut.

\begin{lstlisting}[language=bash]
$ flutter create my_flutter_app
$ cd my_flutter_app
$ tree -L 2
\end{lstlisting}

Contoh keluaran struktur direktori proyek Flutter secara umum ditunjukkan sebagai berikut.

\begin{lstlisting}[language=bash]
my_flutter_app/
|-- android/
|-- ios/
|-- lib/
    |-- main.dart
|-- test/
|-- pubspec.yaml
|-- pubspec.lock
`-- README.md
\end{lstlisting}


Direktori \texttt{android/} berisi seluruh konfigurasi dan kode native yang diperlukan untuk membangun aplikasi Flutter pada platform Android. Di dalam direktori ini terdapat pengaturan Gradle, manifest aplikasi, serta konfigurasi build yang umumnya jarang dimodifikasi kecuali untuk kebutuhan tertentu seperti penambahan permission atau pengaturan build flavor.

Direktori \texttt{ios/} memiliki fungsi yang serupa dengan direktori Android, namun khusus untuk platform iOS. Direktori ini berisi proyek Xcode, konfigurasi signing, serta pengaturan dependensi iOS. Perubahan pada direktori ini biasanya dilakukan ketika mengatur identitas aplikasi, sertifikat, atau integrasi fitur native iOS.

Direktori \texttt{lib/} merupakan inti dari proyek Flutter. Seluruh kode aplikasi Flutter ditulis di dalam direktori ini. Berkas \texttt{main.dart} berfungsi sebagai entry point aplikasi, tempat eksekusi aplikasi Flutter dimulai. Pada proyek berskala besar, direktori \texttt{lib/} umumnya akan dikembangkan menjadi beberapa subdirektori untuk memisahkan fitur, layar, dan logika aplikasi.

Direktori \texttt{test/} digunakan untuk menyimpan seluruh berkas pengujian aplikasi Flutter, termasuk unit test, widget test, dan integration test. Pengujian yang terstruktur dengan baik pada direktori ini berperan penting dalam menjaga kualitas dan stabilitas aplikasi seiring bertambahnya kompleksitas kode.

Berkas \texttt{pubspec.yaml} merupakan berkas konfigurasi utama proyek Flutter yang digunakan untuk mendefinisikan dependensi, aset, versi SDK, serta metadata aplikasi. Sementara itu, \texttt{pubspec.lock} menyimpan versi pasti dari dependensi yang digunakan untuk menjamin konsistensi lingkungan pengembangan.

Dengan memahami struktur direktori proyek Flutter sejak awal, pengembang akan lebih mudah dalam menavigasi kode, menambahkan fitur baru, serta menjaga keteraturan proyek. Struktur ini juga menjadi dasar untuk pengembangan arsitektur aplikasi yang lebih kompleks pada pertemuan-pertemuan selanjutnya.

\subsection{Peran Berkas Utama dalam Proyek Flutter}

Selain memahami struktur direktori secara umum, pengembang Flutter juga perlu mengetahui peran dari berkas-berkas utama yang terdapat di dalam proyek. Setiap berkas memiliki fungsi spesifik yang saling terkait dalam proses pengembangan, build, dan pemeliharaan aplikasi. Pemahaman ini akan membantu pengembang menghindari kesalahan konfigurasi serta mempercepat proses pengembangan aplikasi.

Berkas \texttt{main.dart} merupakan berkas terpenting dalam proyek Flutter karena berfungsi sebagai titik masuk (entry point) aplikasi. Eksekusi aplikasi Flutter selalu dimulai dari fungsi \texttt{main()}, yang kemudian memanggil widget akar aplikasi melalui fungsi \texttt{runApp()}. Seluruh alur awal aplikasi, termasuk inisialisasi state global atau konfigurasi awal, biasanya ditempatkan pada berkas ini atau dipanggil dari berkas ini.

Berkas \texttt{pubspec.yaml} berperan sebagai berkas konfigurasi utama proyek Flutter. Di dalam berkas ini didefinisikan berbagai aspek penting, seperti versi Flutter dan Dart SDK yang digunakan, daftar dependensi eksternal, konfigurasi aset (gambar, ikon, font), serta metadata aplikasi. Setiap kali dependensi ditambahkan atau diubah, berkas ini harus diperbarui dan dependensi diunduh kembali melalui Flutter CLI.

Berkas \texttt{pubspec.lock} secara otomatis dihasilkan oleh Flutter untuk menyimpan versi pasti dari setiap dependensi yang digunakan dalam proyek. Berkas ini memastikan bahwa proyek dapat dibangun secara konsisten pada lingkungan pengembangan yang berbeda, misalnya pada komputer pengembang lain atau pada sistem Continuous Integration (CI). Umumnya berkas ini tidak diedit secara manual oleh pengembang.

Berkas \texttt{README.md} digunakan sebagai dokumentasi singkat proyek. Berkas ini biasanya berisi deskripsi aplikasi, cara menjalankan proyek, serta informasi penting lain yang berguna bagi pengembang lain yang terlibat dalam proyek. Meskipun tidak memengaruhi proses build aplikasi, dokumentasi yang baik sangat membantu dalam kerja tim dan pemeliharaan jangka panjang.

Pada platform Android dan iOS, terdapat pula berkas konfigurasi khusus yang memiliki peran penting. Pada Android, misalnya, berkas konfigurasi Gradle dan Android Manifest menentukan identitas aplikasi, permission, serta pengaturan build. Sementara itu, pada iOS, berkas konfigurasi Xcode dan \texttt{Info.plist} digunakan untuk pengaturan serupa pada platform iOS. Berkas-berkas ini umumnya dimodifikasi hanya ketika diperlukan, seperti saat menambahkan permission atau mengubah identitas aplikasi.

Dengan memahami peran masing-masing berkas utama dalam proyek Flutter, pengembang dapat bekerja secara lebih terstruktur dan efisien. Pengetahuan ini juga menjadi dasar untuk melakukan konfigurasi lanjutan, seperti pengaturan build flavor, integrasi layanan pihak ketiga, serta penerapan arsitektur aplikasi yang lebih kompleks.

\subsection{Konsep Entry Point Aplikasi}

Setiap aplikasi Flutter memiliki satu titik awal eksekusi yang disebut sebagai \textit{entry point}. Entry point merupakan bagian pertama dari kode yang dijalankan ketika aplikasi dimulai, baik pada emulator maupun perangkat fisik. Dalam Flutter, entry point aplikasi secara default didefinisikan melalui fungsi \texttt{main()} yang terdapat pada berkas \texttt{main.dart}.

Fungsi \texttt{main()} berperan sebagai penghubung antara lingkungan runtime Dart dan kerangka kerja Flutter. Ketika aplikasi dijalankan, Dart Virtual Machine akan mengeksekusi fungsi \texttt{main()} terlebih dahulu sebelum Flutter mulai membangun antarmuka pengguna. Oleh karena itu, seluruh proses inisialisasi awal aplikasi biasanya dimulai dari fungsi ini.

Di dalam fungsi \texttt{main()}, aplikasi Flutter dijalankan dengan memanggil fungsi \texttt{runApp()}. Fungsi ini menerima sebuah widget sebagai parameter dan menjadikannya sebagai widget akar (\textit{root widget}) dari seluruh widget tree aplikasi. Widget akar inilah yang kemudian menjadi dasar dalam proses rendering antarmuka pengguna.

\begin{lstlisting}[language=bash]
$ flutter run
\end{lstlisting}

Secara konseptual, alur eksekusi aplikasi Flutter dapat dipahami sebagai berikut: sistem menjalankan fungsi \texttt{main()}, kemudian \texttt{runApp()} dipanggil untuk menginisialisasi widget tree, setelah itu Flutter framework mulai melakukan proses rendering dan pengelolaan siklus hidup widget.

Selain menjalankan aplikasi, fungsi \texttt{main()} juga sering digunakan untuk melakukan konfigurasi awal sebelum antarmuka ditampilkan. Contoh konfigurasi tersebut antara lain adalah inisialisasi dependensi, pengaturan mode aplikasi (development atau production), pengaktifan logging, atau inisialisasi layanan pihak ketiga. Praktik ini memungkinkan aplikasi berada dalam kondisi siap sebelum interaksi pengguna dimulai.

Pada proyek yang lebih kompleks, fungsi \texttt{main()} dapat dikembangkan untuk mendukung berbagai skenario, seperti penggunaan build flavor atau konfigurasi lingkungan yang berbeda. Meskipun demikian, prinsip dasar entry point tetap sama, yaitu menyediakan satu titik awal yang jelas dan terkontrol untuk menjalankan aplikasi Flutter.

Dengan memahami konsep entry point aplikasi, pengembang akan memiliki gambaran yang jelas mengenai bagaimana sebuah aplikasi Flutter dimulai dan bagaimana alur eksekusi awal berlangsung. Pemahaman ini sangat penting sebagai dasar untuk mempelajari topik lanjutan, seperti manajemen state global, dependency injection, serta pengaturan arsitektur aplikasi Flutter secara keseluruhan.


\section{Konfigurasi Proyek dengan \texttt{pubspec.yaml}}
\subsection{Manajemen Dependensi}
\subsection{Pengaturan Asset dan Font}
\subsection{Versi SDK dan Constraint Dependensi}

\section{Manajemen Build dan Flavor Aplikasi}
\subsection{Konsep Build Flavor}
\subsection{Konfigurasi Flavor untuk Development}
\subsection{Konfigurasi Flavor untuk Staging}
\subsection{Konfigurasi Flavor untuk Production}

\section{Flutter CLI dan Alur Pengembangan}
\subsection{Perintah Dasar Flutter CLI}
\subsection{Menjalankan Aplikasi pada Emulator dan Perangkat Fisik}
\subsection{Proses Build Aplikasi}

\section{Ringkasan}

	\chapter{Observabilitas Sistem Terdistribusi}

\section{Tujuan Pembelajaran}

Setelah mengikuti sesi ini, peserta diharapkan mampu:
\begin{itemize}
  \item Memahami observabilitas sebagai pendekatan untuk mengenali dan menalar perilaku sistem berdasarkan data runtime, tanpa bergantung pada struktur internal atau kode sumber sistem.
  \item Mengidentifikasi dan menafsirkan metrik, log, dan trace sebagai representasi perilaku sistem terdistribusi yang melibatkan broker, capturer, transformer, dan web server.
  \item Mengaitkan visualisasi observabilitas pada Prometheus dan Grafana dengan kondisi runtime, beban sistem, serta pengalaman pengguna.
\end{itemize}


\section{Konsep Dasar Observabilitas}

\subsection{Observabilitas vs Monitoring}

Monitoring dan observabilitas sering digunakan secara bergantian, namun keduanya merepresentasikan pendekatan yang berbeda dalam memahami perilaku sistem. Monitoring berfokus pada pengawasan kondisi yang telah diketahui sebelumnya, biasanya melalui sekumpulan metrik atau ambang batas yang telah ditentukan. Pendekatan ini menjawab pertanyaan apakah sistem berada dalam kondisi normal atau tidak.

Observabilitas, di sisi lain, berfokus pada kemampuan untuk menalar kondisi internal sistem berdasarkan sinyal eksternal yang dihasilkan selama sistem berjalan. Tujuannya bukan hanya mendeteksi bahwa terjadi masalah, tetapi juga memahami mengapa dan bagaimana masalah tersebut muncul, termasuk dalam situasi yang sebelumnya tidak diprediksi.

Dalam konteks sistem terdistribusi yang terdiri dari broker, capturer, transformer, dan web server, monitoring umumnya menampilkan indikator terpisah pada setiap komponen, seperti penggunaan sumber daya atau status layanan. Pendekatan ini berguna untuk mendeteksi gangguan lokal, namun sering kali tidak cukup untuk menjelaskan perilaku sistem secara menyeluruh ketika terjadi keterlambatan, penumpukan data, atau kegagalan parsial.

Observabilitas memungkinkan pengamatan perilaku lintas komponen dengan mengaitkan metrik, log, dan trace sebagai satu kesatuan. Melalui korelasi data runtime yang dikumpulkan oleh Prometheus dan divisualisasikan melalui Grafana, peserta dapat menelusuri aliran peristiwa dari satu komponen ke komponen lain, serta memahami dampaknya terhadap respons sistem secara keseluruhan.

Dengan demikian, monitoring dapat dipandang sebagai alat untuk menjawab pertanyaan \emph{``apa yang salah''}, sedangkan observabilitas berperan untuk menjawab \emph{``mengapa hal tersebut terjadi''}. Pendekatan observabilitas menjadi penting dalam sistem modern yang bersifat dinamis, terdistribusi, dan sulit dipahami hanya melalui inspeksi statis atau asumsi desain awal.

\subsection{Metrik, Log, dan Trace}

Metrik, log, dan trace merupakan tiga jenis sinyal utama dalam observabilitas yang merepresentasikan perilaku sistem dari sudut pandang yang berbeda. Ketiganya saling melengkapi dan digunakan bersama untuk membangun pemahaman yang utuh terhadap dinamika sistem saat berjalan.

Metrik menyajikan pengukuran numerik teragregasi yang menggambarkan kondisi sistem dalam rentang waktu tertentu. Contoh metrik meliputi laju pemrosesan, latensi, tingkat kesalahan, dan penggunaan sumber daya. Dalam sistem yang terdiri dari broker, capturer, transformer, dan web server, metrik memungkinkan pengamatan tren seperti peningkatan beban, antrian yang menumpuk, atau penurunan throughput tanpa harus melihat peristiwa individual.

Log merekam peristiwa diskret yang terjadi selama eksekusi sistem, biasanya dalam bentuk pesan berurutan yang merefleksikan aktivitas internal komponen. Log memberikan konteks yang lebih kaya dibandingkan metrik, seperti urutan kejadian, kondisi tertentu, atau keputusan yang diambil oleh sistem. Pada sistem terdistribusi, log membantu menjelaskan apa yang sedang dilakukan oleh masing-masing komponen pada saat tertentu, terutama ketika terjadi kegagalan atau perilaku tidak normal.

Trace merepresentasikan alur eksekusi suatu permintaan atau data saat melewati beberapa komponen sistem. Melalui trace, satu peristiwa dapat ditelusuri dari awal hingga akhir, misalnya dari capturer ke broker, diteruskan ke transformer, hingga akhirnya disajikan oleh web server. Trace memberikan gambaran hubungan sebab-akibat antar komponen serta memungkinkan identifikasi titik keterlambatan atau bottleneck dalam alur sistem.

Dalam praktik observabilitas, ketiga sinyal ini tidak digunakan secara terpisah. Metrik sering digunakan untuk mendeteksi adanya anomali, log digunakan untuk memberikan konteks terhadap anomali tersebut, dan trace digunakan untuk menelusuri jalur eksekusi yang menyebabkan kondisi tersebut muncul. Kombinasi ini memungkinkan pemahaman perilaku sistem secara menyeluruh, bahkan ketika masalah yang terjadi tidak pernah didefinisikan sebelumnya.

Dengan memanfaatkan metrik, log, dan trace secara bersamaan, observabilitas memungkinkan peserta untuk berpindah dari pengamatan permukaan menuju pemahaman mendalam terhadap dinamika runtime sistem terdistribusi.


\subsection{Observabilitas sebagai Pendekatan Bottom-Up}

Pendekatan bottom-up dalam observabilitas menekankan pemahaman sistem yang dimulai dari perilaku nyata yang muncul selama eksekusi, bukan dari asumsi desain atau spesifikasi awal. Dalam pendekatan ini, sistem dipahami melalui sinyal runtime yang dihasilkannya, seperti metrik, log, dan trace, yang merefleksikan interaksi aktual antar komponen.

Berbeda dengan pendekatan top-down yang berangkat dari arsitektur, diagram, atau dokumentasi desain, pendekatan bottom-up mengajak peserta untuk terlebih dahulu mengamati apa yang benar-benar terjadi ketika sistem dijalankan. Hal ini menjadi penting terutama pada sistem terdistribusi yang kompleks, di mana perilaku aktual sering kali menyimpang dari ekspektasi desain akibat beban, kegagalan parsial, atau interaksi non-linear antar komponen.

Dalam sistem yang melibatkan broker, capturer, transformer, dan web server, pendekatan bottom-up memungkinkan pengamatan aliran data dan peristiwa tanpa harus memahami implementasi internal masing-masing komponen. Peserta dapat mulai dengan mengamati metrik seperti laju data masuk, latensi pemrosesan, atau tingkat kesalahan, kemudian menelusuri log dan trace untuk memahami hubungan sebab-akibat di balik perubahan yang teramati.

Pendekatan ini juga mendorong eksplorasi sistem secara iteratif. Pengamatan awal terhadap dashboard atau grafik metrik dapat memunculkan pertanyaan baru, yang selanjutnya dijawab dengan memperkaya konteks melalui log dan trace. Proses ini membentuk siklus observasi dan penalaran yang berulang, di mana pemahaman sistem berkembang seiring dengan meningkatnya kedalaman pengamatan.

Dengan menempatkan data runtime sebagai sumber utama pengetahuan, observabilitas sebagai pendekatan bottom-up membantu membangun intuisi tentang perilaku sistem yang bersifat dinamis dan kontekstual. Pendekatan ini sangat relevan untuk sistem modern yang terus berevolusi, berskala besar, dan tidak dapat sepenuhnya dipahami hanya melalui analisis statis atau dokumentasi desain.




\section{Arsitektur Sistem Studi Kasus}


\begin{figure}[htbp]
\centering
\scalebox{0.9}{
\begin{tikzpicture}[
  font=\small\bfseries,
  node distance=9mm and 12mm,
  box/.style={
    draw,
    rounded corners=2mm,
    align=center,
    minimum width=32mm,
    minimum height=9mm,
    fill=blue!10
  },
  brokerbox/.style={box,fill=orange!15},
  webbox/.style={box,fill=green!12},
  browserbox/.style={box,fill=gray!12},
  extbox/.style={box,fill=purple!10},
  arrow/.style={-Latex,thick},
  dottedarrow/.style={-Latex,thick,dotted}
]

% Core nodes (note: Prometheus service embedded in each node)
\node[box ] (transformer) {Transformer\\(Node B)\\{\scriptsize Prometheus service}};
\node[box,left=of transformer, xshift=-10mm] (capturer) {Capturer\\(Node A)\\{\scriptsize Prometheus service}};
\node[brokerbox, below=of capturer, yshift=-5mm] (broker) {Broker\\(XSUB / XPUB)\\{\scriptsize Prometheus service}};

\node[browserbox,below=of broker] (browser) {Browser};
\node[webbox,right=of browser, xshift=10mm](web) {Web Server\\(Node C)\\{\scriptsize Prometheus service}};

% External observability nodes
\node[extbox,right=of web] (grafana) {Grafana};
\node[extbox,above=of grafana, yshift=10mm] (prom) {Prometheus};

% Data flow arrows
\draw[arrow] (capturer) -- node[above, xshift=-8mm]{PUB \texttt{raw}} (broker);
\draw[arrow] (broker) -- node[above, yshift=-5mm, xshift=8mm]{SUB \texttt{raw}} (transformer);
\draw[arrow] (transformer.south) |- node[below]{PUB \texttt{processed}} (broker.east);
\draw[arrow] (broker) -- node[right,xshift=3mm]{SUB \texttt{processed}} (web.north);
\draw[arrow] (web.west) -- node[above]{HTTP MJPEG} (browser.east);

% Observability (scraping + visualization)
\draw[dottedarrow] (capturer.south) -- (prom);
\draw[dottedarrow] (broker) -- (prom);
\draw[dottedarrow] (transformer.east) -- (prom);
\draw[dottedarrow] (web.east) -- (prom);
\draw[arrow] (prom) -- (grafana);

\end{tikzpicture}
}
\caption{Arsitektur aliran data menggunakan komunikasi publish--subscribe berbasis broker, dengan layanan Prometheus tertanam pada setiap node serta Prometheus dan Grafana sebagai komponen observabilitas eksternal}
\label{fig:broker-pubsub-observability}
\end{figure}

Pada Gambar~\ref{fig:broker-pubsub-observability} ditunjukkan arsitektur sistem studi kasus yang digunakan untuk mengeksplorasi perilaku sistem terdistribusi berbasis komunikasi \emph{publish--subscribe}. Arsitektur ini terdiri dari beberapa node utama yang berinteraksi melalui sebuah \emph{message broker}, serta dilengkapi dengan mekanisme observabilitas yang terintegrasi.

Node \textbf{Capturer (Node A)} berperan sebagai sumber data awal. Node ini menghasilkan data mentah (\emph{raw data}) dan mempublikasikannya ke broker menggunakan mekanisme \emph{publish}. Dengan pendekatan ini, produsen data tidak berkomunikasi langsung dengan konsumen, melainkan melalui perantara.

Node \textbf{Broker} berfungsi sebagai pusat pertukaran data dengan pola komunikasi \emph{XSUB/XPUB}. Broker menerima data mentah dari capturer dan mendistribusikannya kepada node yang melakukan \emph{subscribe}. Pola ini memungkinkan pemisahan yang jelas antara pengirim dan penerima data, serta mendukung skalabilitas sistem.

Node \textbf{Transformer (Node B)} melakukan \emph{subscribe} terhadap data mentah dari broker untuk kemudian memprosesnya. Data hasil pemrosesan (\emph{processed data}) dipublikasikan kembali ke broker, sehingga dapat digunakan oleh komponen lain tanpa ketergantungan langsung.

Node \textbf{Web Server (Node C)} melakukan \emph{subscribe} terhadap data hasil pemrosesan dari broker dan menyajikannya kepada pengguna. Penyajian data dilakukan melalui antarmuka web menggunakan protokol HTTP dalam bentuk aliran MJPEG.

Interaksi pengguna direpresentasikan oleh \textbf{Browser}, yang mengakses web server untuk melihat hasil akhir pemrosesan. Dari sudut pandang pengguna, sistem tampak sebagai layanan web, meskipun terdiri dari beberapa komponen terdistribusi di belakang layar.

Setiap node aplikasi dilengkapi dengan \textbf{layanan Prometheus yang tertanam} untuk mengekspor metrik runtime. Metrik-metrik ini mencerminkan perilaku aktual sistem selama eksekusi. Prometheus eksternal bertugas mengumpulkan metrik dari seluruh node melalui mekanisme \emph{scraping}.

Data observabilitas yang telah dikumpulkan selanjutnya divisualisasikan melalui \textbf{Grafana}. Dashboard Grafana memungkinkan pemantauan dan analisis perilaku sistem secara menyeluruh, serta mendukung pendekatan observabilitas berbasis pengamatan langsung terhadap data runtime.

Arsitektur ini dirancang untuk mendukung pendekatan observabilitas \emph{bottom-up}, di mana pemahaman sistem dibangun dari perilaku nyata yang teramati selama sistem berjalan, sebelum dilakukan analisis struktur atau optimasi desain.


\section{Prometheus}

Prometheus merupakan sistem pemantauan dan pengumpulan metrik yang dirancang untuk sistem terdistribusi dan dinamis. Dalam studi kasus ini, Prometheus digunakan sebagai komponen observabilitas terpusat yang mengumpulkan metrik runtime dari setiap node aplikasi, yaitu Capturer, Broker, Transformer, dan Web Server.

Prometheus menerapkan model \emph{pull-based}, di mana server Prometheus secara periodik melakukan pengambilan (\emph{scraping}) metrik dari endpoint yang diekspos oleh masing-masing node. Setiap node aplikasi menjalankan layanan Prometheus tertanam yang mengekspor metrik runtime tanpa mengganggu alur pemrosesan data utama.

Metrik yang dikumpulkan disimpan dalam bentuk \emph{time-series data}, sehingga memungkinkan analisis perilaku sistem sepanjang waktu, termasuk pengamatan tren, lonjakan beban, latensi, serta indikasi kegagalan parsial. Dalam arsitektur ini, Prometheus tidak terlibat langsung dalam alur data aplikasi, melainkan berfungsi murni sebagai mekanisme observasi.

Pada demo ini, Prometheus tidak diinstal secara manual pada sistem operasi host. Prometheus dijalankan sebagai \emph{container} menggunakan \emph{Docker image} resmi yang di-\emph{pull} dari repositori. Pendekatan ini memberikan lingkungan runtime yang konsisten, mudah direproduksi, dan terisolasi dari node aplikasi.

Dengan menjalankan Prometheus sebagai container, konfigurasi scraping dan penyimpanan metrik dapat dikelola secara terpisah dari kode aplikasi. Selain itu, Prometheus dapat dengan mudah dijalankan, dihentikan, atau dikonfigurasi ulang tanpa memengaruhi proses utama sistem studi kasus.

Untuk mendukung ekspor metrik dari aplikasi berbasis Python, digunakan library \texttt{prometheus-client}. Library ini memungkinkan setiap node menyediakan endpoint metrik yang dapat diakses oleh Prometheus eksternal.

\begin{lstlisting}[language=bash]
pip install prometheus-client
\end{lstlisting}

Dengan pendekatan ini, setiap node aplikasi mengekspor sinyal runtime secara mandiri, sementara Prometheus berperan sebagai pengumpul data terpusat yang berjalan sebagai layanan terpisah. Pemisahan peran ini mendukung observabilitas berbasis data runtime dan memungkinkan analisis perilaku sistem secara \emph{bottom-up} sebelum dilakukan penalaran terhadap struktur atau desain sistem.



\section{Grafana}

Grafana merupakan platform visualisasi data yang digunakan untuk menampilkan dan menganalisis metrik yang dikumpulkan oleh Prometheus. Dalam studi kasus ini, Grafana berperan sebagai lapisan presentasi yang memungkinkan pengamatan perilaku sistem secara intuitif melalui \emph{dashboard}.

Grafana tidak mengumpulkan data secara langsung dari node aplikasi. Grafana berfungsi sebagai klien yang melakukan kueri ke Prometheus untuk mengambil data metrik yang telah tersimpan dalam bentuk \emph{time-series}. Pemisahan peran ini memastikan bahwa visualisasi tidak memengaruhi mekanisme pengumpulan maupun pemrosesan data utama sistem.

Pada demo ini, Grafana tidak dipasang secara manual di sistem operasi host. Sebaliknya, Grafana dijalankan sebagai \emph{container} dengan memanfaatkan \emph{Docker image} resmi yang di-\emph{pull} dari repositori. Pendekatan ini memberikan lingkungan runtime yang konsisten, dapat direproduksi, dan mudah dikonfigurasi ulang, terutama dalam konteks eksperimen dan pembelajaran.

Dengan menjalankan Grafana sebagai container, proses penyebaran dan penghentian layanan visualisasi dapat dilakukan tanpa memengaruhi node aplikasi maupun server Prometheus. Selain itu, konfigurasi Grafana, seperti koneksi ke Prometheus dan definisi dashboard, dapat dikelola secara terpisah dari kode aplikasi.

Melalui dashboard Grafana, metrik dari berbagai node seperti Capturer, Broker, Transformer, dan Web Server dapat ditampilkan secara bersamaan. Visualisasi ini memungkinkan perbandingan perilaku antar node, identifikasi pola beban, serta pengamatan perubahan kondisi sistem dari waktu ke waktu.

Dalam konteks pembelajaran, penggunaan Grafana sebagai container juga memperkuat pemahaman bahwa komponen observabilitas dapat diperlakukan sebagai layanan terpisah. Peserta dapat fokus pada interpretasi perilaku sistem melalui dashboard tanpa harus berurusan dengan kompleksitas instalasi perangkat lunak secara manual.

Dengan demikian, Grafana berfungsi sebagai lapisan visualisasi observabilitas yang fleksibel, terisolasi, dan mudah direproduksi, sejalan dengan tujuan demo untuk menekankan observabilitas berbasis data runtime.


\section{OpenTelemetry}

OpenTelemetry merupakan standar terbuka untuk pengumpulan dan representasi telemetri yang mencakup metrik, log, dan trace. Dalam konteks studi kasus ini, OpenTelemetry diposisikan sebagai kerangka konseptual yang menyatukan berbagai jenis sinyal observabilitas, terutama untuk memahami alur eksekusi lintas node dalam sistem terdistribusi.

Berbeda dengan Prometheus yang berfokus pada pengumpulan metrik berbasis \emph{pull}, OpenTelemetry menekankan konsistensi semantik dalam pembangkitan dan propagasi data telemetri. OpenTelemetry menyediakan model data dan mekanisme instrumentasi yang memungkinkan peristiwa runtime direpresentasikan secara seragam, terlepas dari bahasa pemrograman atau platform yang digunakan.

Dalam arsitektur studi kasus ini, OpenTelemetry relevan terutama untuk konsep \emph{distributed tracing}. Melalui trace, satu alur pemrosesan data dapat diikuti sejak data dihasilkan oleh Capturer, diteruskan melalui Broker, diproses oleh Transformer, hingga akhirnya disajikan oleh Web Server. Setiap tahapan direpresentasikan sebagai \emph{span} yang saling terhubung dalam satu konteks trace.

Penggunaan OpenTelemetry membantu mengaitkan metrik kuantitatif dengan konteks eksekusi yang lebih kaya. Ketika terjadi lonjakan latensi atau penurunan throughput yang terdeteksi melalui metrik Prometheus, trace yang dihasilkan melalui OpenTelemetry dapat digunakan untuk menelusuri titik keterlambatan atau bottleneck pada node tertentu.

Dalam demo ini, OpenTelemetry tidak selalu diaktifkan secara penuh sebagai sistem terpisah, melainkan diperkenalkan sebagai kerangka konseptual untuk memahami relasi antara metrik, log, dan trace. Pendekatan ini menekankan bahwa observabilitas modern tidak hanya bergantung pada satu jenis sinyal, tetapi pada kombinasi berbagai bentuk telemetri yang saling melengkapi.

Dengan memperkenalkan OpenTelemetry, peserta didorong untuk memahami observabilitas sebagai praktik lintas lapisan, yang menghubungkan sinyal runtime dengan alur eksekusi sistem terdistribusi. Pemahaman ini memperkuat pendekatan observabilitas \emph{bottom-up}, di mana sistem dianalisis melalui jejak perilaku aktual yang muncul selama eksekusi.



\section{Eksperimen Observabilitas}

Bagian ini menjelaskan eksperimen observabilitas yang dilakukan menggunakan sistem studi kasus berbasis arsitektur broker--capturer--transformer--web server. Seluruh kode sumber eksperimen tersedia secara terbuka pada repositori berikut:

\noindent\url{https://github.com/alfa-yohannis/advanced-software-engineering/tree/main/projects/session-02/python}

Eksperimen ini bertujuan untuk memperlihatkan bagaimana perilaku sistem terdistribusi dapat diamati melalui data runtime, tanpa harus melakukan analisis kode atau desain secara langsung. Fokus utama eksperimen adalah integrasi metrik, orkestrasi layanan observabilitas, serta interpretasi hasil pengamatan.

\subsection{Tahapan Eksperimen}

Eksperimen observabilitas dilakukan melalui beberapa tahapan berikut:

\begin{enumerate}
  \item \textbf{Instalasi dan Konfigurasi Prometheus}  

Prometheus ditambahkan sebagai komponen observabilitas eksternal yang bertugas mengumpulkan metrik runtime dari seluruh node aplikasi. Prometheus dijalankan sebagai container dan dikonfigurasi untuk melakukan \emph{scraping} terhadap endpoint metrik yang diekspor oleh masing-masing node.

Untuk memungkinkan setiap node aplikasi berbasis Python mengekspor metrik, digunakan library \texttt{prometheus-client}. Library ini dipasang pada seluruh node aplikasi, baik Capturer, Broker, Transformer, maupun Web Server.

\begin{lstlisting}[language=bash]
pip install prometheus-client
\end{lstlisting}

Setelah library terpasang, setiap node menyediakan endpoint \texttt{/metrics} yang berisi metrik runtime dan dapat diakses oleh Prometheus eksternal. Endpoint ini tidak memengaruhi alur data utama sistem dan hanya berfungsi sebagai sumber observasi.

Prometheus dijalankan sebagai layanan terpisah menggunakan Docker. Konfigurasi ini memungkinkan Prometheus mengumpulkan metrik dari seluruh node tanpa harus diinstal langsung pada host sistem.


Pendekatan ini memastikan bahwa mekanisme observabilitas terisolasi dari alur data utama aplikasi, namun tetap mampu merekam perilaku sistem secara menyeluruh untuk keperluan analisis dan eksperimen observabilitas.


  \item \textbf{Pembaruan Kode pada Seluruh Node Aplikasi}  

Kode pada node Capturer, Broker, Transformer, dan Web Server diperbarui untuk mengekspor metrik runtime menggunakan library \texttt{prometheus-client}. Pola implementasi yang digunakan pada demo ini adalah menjalankan HTTP endpoint \texttt{/metrics} pada port yang berbeda untuk setiap node, sehingga Prometheus eksternal dapat melakukan \emph{scraping} secara terpisah per komponen.

Selain itu, ditambahkan \emph{shared library} metrik untuk memastikan konsistensi definisi metrik antar node, misalnya konsistensi penamaan metrik, label yang digunakan, serta tipe metrik (\texttt{Counter}, \texttt{Gauge}, \texttt{Histogram}). Shared library ini bertujuan mencegah duplikasi definisi metrik dan mengurangi risiko inkonsistensi antar node.

Berikut contoh struktur minimal shared library metrik (misal: \texttt{shared/observability.py}):

\begin{lstlisting}[style=PythonStyle]
# shared/observability.py
from prometheus_client import Counter, Gauge, Histogram

# Throughput / volume counters
frames_in_total = Counter(
    "frames_in_total",
    "Total frames received/consumed by a node",
    ["service"]
)

frames_out_total = Counter(
    "frames_out_total",
    "Total frames produced/published by a node",
    ["service"]
)

# Runtime gauges
queue_size = Gauge(
    "queue_size",
    "Approximate queue size (if applicable)",
    ["service", "queue"]
)

# Latency histograms (seconds)
processing_seconds = Histogram(
    "processing_seconds",
    "Processing latency per frame/message in seconds",
    ["service", "stage"]
)
\end{lstlisting}

Setiap node kemudian mengimpor shared library ini dan menjalankan server metrik. Contoh berikut menunjukkan pola umum yang dapat digunakan pada seluruh node:

\begin{lstlisting}[style=PythonStyle]
# contoh potongan umum pada node (capturer/broker/transformer/web)
from prometheus_client import start_http_server
from shared.metrics2 import frames_in_total, frames_out_total, processing_seconds

SERVICE = "capturer"  # ganti sesuai node: broker/transformer/web_server

def init_metrics_server(metrics_port: int) -> None:
    # Menjalankan endpoint /metrics pada port tertentu
    start_http_server(metrics_port)

# Panggil sekali pada startup node
init_metrics_server(9101)  # contoh untuk capturer
\end{lstlisting}

Dengan pola ini, ekspor metrik menjadi seragam: setiap node menjalankan endpoint \texttt{/metrics} dan menggunakan definisi metrik yang sama melalui shared library.

\item \textbf{Penambahan Metrik Runtime}  

Setiap node dilengkapi dengan metrik tambahan yang merepresentasikan perilaku utama sistem, seperti jumlah frame yang diproses, laju publikasi data, serta aktivitas pemrosesan dan penyajian data. Metrik dipilih agar mudah dikaitkan dengan fenomena runtime pada sistem terdistribusi, misalnya throughput, latensi, serta indikasi penumpukan (backpressure) atau bottleneck.

Berikut contoh penambahan metrik pada masing-masing node (cuplikan minimal):

\begin{enumerate}
  \item \textbf{Capturer (Node A):} menghitung frame yang dihasilkan dan waktu akuisisi/publish.
\begin{lstlisting}[style=PythonStyle]
# di loop utama capturer
from time import time
from shared.metrics2 import frames_out_total, processing_seconds

t0 = time()
# ... capture frame ...
# ... publish raw frame ...
frames_out_total.labels(service="capturer").inc()
processing_seconds.labels(service="capturer", stage="capture_publish").observe(time() - t0)
\end{lstlisting}

  \item \textbf{Broker:} menghitung pesan yang diteruskan (raw/processed) dan (opsional) ukuran antrean internal jika ada.
\begin{lstlisting}[style=PythonStyle]
from shared.metrics2 import frames_in_total, frames_out_total

# saat menerima raw dari capturer
frames_in_total.labels(service="broker").inc()

# saat meneruskan ke subscriber (raw atau processed)
frames_out_total.labels(service="broker").inc()
\end{lstlisting}

  \item \textbf{Transformer (Node B):} menghitung frame yang dikonsumsi dan yang diproduksi, serta latensi pemrosesan transformasi.
\begin{lstlisting}[style=PythonStyle]
from time import time
from shared.metrics2 import frames_in_total, frames_out_total, processing_seconds

# saat menerima raw
frames_in_total.labels(service="transformer").inc()

t0 = time()
# ... transform frame ...
frames_out_total.labels(service="transformer").inc()
processing_seconds.labels(service="transformer", stage="transform").observe(time() - t0)
\end{lstlisting}

  \item \textbf{Web Server (Node C):} menghitung frame yang disajikan dan (opsional) latensi penyajian/streaming.
\begin{lstlisting}[style=PythonStyle]
from shared.metrics2 import frames_in_total, frames_out_total

# saat menerima processed dari broker
frames_in_total.labels(service="web_server").inc()

# saat frame dipublikasikan ke stream MJPEG (per frame terkirim)
frames_out_total.labels(service="web_server").inc()
\end{lstlisting}
\end{enumerate}

Dengan penambahan metrik tersebut, peserta dapat melakukan pengamatan lintas node: misalnya membandingkan laju \texttt{frames\_out\_total} pada Capturer dengan \texttt{frames\_in\_total} pada Transformer, atau mengamati distribusi \texttt{processing\_seconds} untuk mengidentifikasi bottleneck pemrosesan. Kombinasi metrik ini juga memudahkan korelasi antara gejala yang terlihat di dashboard dengan peristiwa runtime pada sistem.


  \item \textbf{Penambahan Node Prometheus dan Grafana serta Konfigurasi Docker Compose}  

Dua node baru ditambahkan ke dalam arsitektur sistem, yaitu \textbf{Prometheus} sebagai pengumpul metrik dan \textbf{Grafana} sebagai lapisan visualisasi. Keduanya dijalankan sebagai container terpisah dari node aplikasi, sehingga mekanisme observabilitas tetap terisolasi dari alur pemrosesan data utama.

Berkas \texttt{docker-compose.yml} diperbarui untuk menyertakan layanan Prometheus dan Grafana, serta mendefinisikan relasi jaringan antar seluruh container. Dengan pendekatan ini, seluruh sistem (node aplikasi + observabilitas) dapat dijalankan secara terorkestrasi melalui satu perintah.


Berikut cuplikan konfigurasi layanan \texttt{prometheus} dan \texttt{grafana} pada \texttt{docker-compose.yml}. Prometheus memuat konfigurasi \texttt{prometheus.yml} melalui \emph{volume mount}, sedangkan Grafana disiapkan untuk berjalan sebagai container visualisasi.

\begin{lstlisting}[language=bash]
services:
  prometheus:
    image: prom/prometheus:latest
    container_name: prometheus
    ports:
      - "9090:9090"
    volumes:
      - ./prometheus/prometheus.yml:/etc/prometheus/prometheus.yml:ro
    command:
      - "--config.file=/etc/prometheus/prometheus.yml"
    depends_on:
      - capturer
      - broker
      - transformer
      - web

  grafana:
    image: grafana/grafana:latest
    container_name: grafana
    ports:
      - "3000:3000"
    depends_on:
      - prometheus
\end{lstlisting}

Konfigurasi di atas mengasumsikan node aplikasi (capturer, broker, transformer, web\_server) telah didefinisikan sebagai layanan lain dalam file \texttt{docker-compose.yml} yang sama, serta setiap node telah mengekspor endpoint \texttt{/metrics} pada port masing-masing.


Berkas \texttt{prometheus.yml} mendefinisikan interval scraping serta daftar target endpoint metrik dari seluruh node aplikasi. Karena Prometheus berjalan di dalam jaringan Docker Compose, target diakses menggunakan \textbf{nama service} (bukan \texttt{localhost}).

\begin{lstlisting}[language=bash]
global:
  scrape_interval: 5s
  evaluation_interval: 5s

scrape_configs:
  - job_name: "capturer"
    static_configs:
      - targets: ["capturer:9101"]

  - job_name: "broker"
    static_configs:
      - targets: ["broker:9102"]

  - job_name: "transformer"
    static_configs:
      - targets: ["transformer:9103"]

  - job_name: "web_server"
    static_configs:
      - targets: ["web:9104"]
\end{lstlisting}

Dengan konfigurasi ini, Prometheus secara periodik mengumpulkan metrik dari endpoint masing-masing node dan menyimpannya sebagai \emph{time-series}. Pengumpulan metrik dilakukan tanpa mengganggu alur data utama sistem.



Grafana dikonfigurasi agar terhubung ke Prometheus sebagai \emph{data source}. Karena Grafana berjalan dalam jaringan Docker Compose yang sama, alamat yang digunakan adalah \texttt{http://prometheus:9090} (nama service Prometheus di dalam jaringan Docker), bukan \texttt{http://localhost:9090}.

Konfigurasi data source dapat dilakukan melalui antarmuka Grafana:
\begin{enumerate}
  \item Buka Grafana pada \url{http://localhost:3000}.
  \item Masuk menggunakan kredensial default (umumnya \texttt{admin/admin}, kemudian diminta mengganti kata sandi).
  \item Pilih \texttt{Connections} $\rightarrow$ \texttt{Data sources} $\rightarrow$ \texttt{Add data source}.
  \item Pilih tipe \texttt{Prometheus}.
  \item Isi URL menjadi \texttt{http://prometheus:9090}.
  \item Klik \texttt{Save \& Test} untuk memastikan koneksi berhasil.
\end{enumerate}

Setelah data source aktif, metrik yang telah dikumpulkan Prometheus dapat divisualisasikan langsung melalui dashboard Grafana, sehingga peserta dapat melakukan observasi perilaku sistem secara \emph{bottom-up} berdasarkan data runtime.

\end{enumerate}

\subsection{Titik Observasi dan Antarmuka Sistem}

Setelah seluruh layanan dijalankan, eksperimen observabilitas dapat diamati melalui beberapa endpoint berikut:

\begin{itemize}
  \item \textbf{Antarmuka Utama Aplikasi}
    \begin{itemize}
      \item \url{http://localhost:8000/}  
            Menampilkan antarmuka web utama.
      \item \url{http://localhost:8000/stream.mjpg}  
            Menyediakan aliran video MJPEG hasil pemrosesan sistem.
    \end{itemize}

  \item \textbf{Endpoint Metrik Runtime}
    \begin{itemize}
      \item Capturer: \url{http://localhost:9101/metrics}
      \item Broker: \url{http://localhost:9102/metrics}
      \item Transformer: \url{http://localhost:9103/metrics}
      \item Web Server: \url{http://localhost:9104/metrics}
    \end{itemize}

  \item \textbf{Layanan Observabilitas}
    \begin{itemize}
      \item Prometheus: \url{http://localhost:9090}  
            Digunakan untuk melakukan kueri dan eksplorasi metrik secara langsung.
      \item Grafana: \url{http://localhost:3000}  
            Digunakan untuk memvisualisasikan metrik melalui dashboard.
    \end{itemize}
\end{itemize}

Melalui eksperimen ini, peserta dapat mengamati bagaimana perubahan beban, alur data, dan aktivitas pemrosesan pada masing-masing node tercermin secara langsung pada metrik dan visualisasi. Pendekatan ini menegaskan peran observabilitas sebagai sarana utama untuk memahami perilaku sistem secara \emph{bottom-up} berdasarkan data runtime.


\section{Hasil Eksperimen}

Bagian ini menyajikan hasil eksperimen observabilitas yang diperoleh dari sistem studi kasus berbasis pipeline Capturer--Broker--Transformer--Web Server. Hasil eksperimen dirangkum dalam dua tabel utama yang merepresentasikan karakteristik komponen pipeline dan dampak peningkatan jumlah klien pada sisi penyajian data.

\subsection*{Analisis Komponen Pipeline}


\begin{table}[htbp]
\centering
\caption{Ringkasan Peran dan Beban Kerja Setiap Komponen Sistem}
\label{tab:pipeline-comparison-simple}
\begin{tabular}{|
p{0.14\linewidth}|
p{0.25\linewidth}|
p{0.16\linewidth}|
p{0.14\linewidth}|
p{0.14\linewidth}|
}
\hline
\textbf{Komponen} &
\textbf{Fungsi Utama} &
\textbf{Jumlah Data Diproses} &
\textbf{Beban CPU} &
\textbf{Penggunaan Memori} \\
\hline
Capturer &
Mengambil video dan mengubahnya menjadi gambar JPEG &
8.473 gambar dihasilkan &
Tinggi (\textasciitilde 54\%) &
Sedang (\textasciitilde 128 MB) \\
\hline
Broker &
Meneruskan data antar komponen (tanpa memproses isi data) &
17.030 pesan diteruskan &
Sangat rendah (\textasciitilde 2\%) &
Rendah (\textasciitilde 32 MB) \\
\hline
Transformer &
Mengolah gambar (decode, ubah ke grayscale, encode ulang) &
8.641 gambar masuk dan keluar &
Sedang (\textasciitilde 21\%) &
Sedang (45--52 MB) \\
\hline
Web Server &
Menyajikan hasil akhir ke pengguna melalui web &
8.819 gambar dikirim ke pengguna &
Sangat rendah (\textasciitilde 0\%) &
Tinggi (\textasciitilde 193 MB) \\
\hline
\end{tabular}
\end{table}


Tabel~\ref{tab:pipeline-comparison-simple} menyajikan ringkasan dan perbandingan karakteristik setiap komponen dalam pipeline pemrosesan data. Setiap komponen menunjukkan pola perilaku runtime yang berbeda sesuai dengan peran fungsionalnya.

Node \textbf{Capturer} berperan sebagai sumber data utama, bertanggung jawab atas pengambilan video dan pengkodean JPEG. Aktivitas ini tercermin dari penggunaan CPU yang relatif tinggi dibandingkan komponen lain, serta konsumsi memori yang stabil. Jumlah frame keluaran menunjukkan laju produksi data yang konsisten selama eksperimen.

Node \textbf{Broker} berfungsi sebagai perantara komunikasi berbasis XSUB/XPUB. Meskipun broker menangani jumlah frame yang besar, penggunaan CPU dan memori relatif rendah. Hal ini menunjukkan bahwa broker terutama melakukan \emph{message forwarding} tanpa operasi pemrosesan data yang berat, sehingga overhead komputasinya minimal.

Node \textbf{Transformer} melakukan operasi pemrosesan yang lebih kompleks, yaitu dekode JPEG, transformasi citra (grayscale), dan enkode ulang. Aktivitas ini tercermin dari penggunaan CPU yang lebih tinggi dibandingkan broker, serta konsumsi memori yang moderat. Jumlah frame masuk dan keluar yang seimbang menunjukkan bahwa tidak terjadi kehilangan data selama proses transformasi.

Node \textbf{Web Server} berfungsi sebagai komponen penyaji hasil akhir melalui HTTP/MJPEG. Konsumsi memori relatif lebih tinggi dibandingkan komponen lain, yang dapat dikaitkan dengan pengelolaan buffer streaming dan koneksi klien. Penggunaan CPU yang rendah pada saat \emph{scraping} metrik menunjukkan bahwa beban utama web server tidak selalu tercermin secara langsung dalam snapshot CPU tunggal, melainkan dipengaruhi oleh pola akses klien.

Secara keseluruhan, tabel ini menunjukkan bahwa observabilitas memungkinkan pemetaan peran fungsional setiap komponen ke karakteristik penggunaan sumber daya yang terukur.

\subsection*{Dampak Jumlah Klien MJPEG}


\begin{table}[htbp]
\centering
\caption{Performa Web Server saat Jumlah Pengguna Bertambah}
\label{tab:webserver_clients_simple}
\begin{tabular}{|
p{0.25\linewidth}|
p{0.15\linewidth}|
p{0.15\linewidth}|
p{0.25\linewidth}|
}
\hline
\textbf{Yang Diukur} &
\textbf{1 Pengguna} &
\textbf{6 Pengguna} &
\textbf{Perubahan} \\
\hline
Jumlah pengguna yang terhubung &
1 &
6 &
Naik 6 kali \\
\hline
Jumlah gambar yang diterima server &
7.070 &
7.517 &
Naik sedikit (+6\%) \\
\hline
Total data yang diterima (MB) &
697 &
741 &
Naik sedikit (+6\%) \\
\hline
Ukuran rata-rata gambar (KB) &
98,6 &
98,6 &
Tidak berubah \\
\hline
Waktu rata-rata gambar sampai ke pengguna (ms) &
30,1 &
30,2 &
Hampir sama \\
\hline
Waktu terlama yang dirasakan pengguna (ms) &
$\le$ 100 &
$\le$ 100 &
Tidak berubah \\
\hline
Penggunaan CPU server (\%) &
1 &
4 &
Naik sedikit \\
\hline
Penggunaan memori server (MB) &
133 &
157 &
Naik 24 MB \\
\hline
Jumlah koneksi/file yang terbuka &
18 &
23 &
Naik 5 \\
\hline
\end{tabular}
\end{table}


Tabel~\ref{tab:webserver_clients_simple} menyajikan perbandingan performa Web Server ketika jumlah klien MJPEG meningkat dari satu menjadi enam browser secara bersamaan. Hasil ini digunakan untuk mengevaluasi skalabilitas dan stabilitas sistem pada sisi penyajian data.

Perlu dicatat bahwa pengambilan cuplikan data untuk skenario enam pengguna dilakukan pada waktu yang sedikit lebih belakangan dibandingkan skenario satu pengguna. Akibatnya, jumlah gambar yang diterima serta total data yang diterima pada skenario enam pengguna tercatat lebih besar. Perbedaan ini mencerminkan durasi pengamatan yang lebih panjang, bukan semata-mata peningkatan laju pemrosesan atau perubahan perilaku sistem.

Peningkatan jumlah klien dari satu menjadi enam tidak menyebabkan perubahan signifikan pada ukuran rata-rata JPEG maupun latensi end-to-end. Nilai latensi rata-rata dan batas atas latensi tetap berada dalam rentang yang sama, menunjukkan bahwa sistem mampu mempertahankan kualitas layanan meskipun jumlah klien meningkat.

Dari sisi sumber daya, peningkatan jumlah klien menyebabkan kenaikan penggunaan CPU dan memori pada Web Server, serta bertambahnya jumlah \emph{file descriptor} terbuka. Namun, peningkatan tersebut bersifat linier dan masih dalam batas yang dapat diterima untuk skenario eksperimen ini. Tidak terlihat adanya gejala degradasi performa yang signifikan atau bottleneck kritis.

Hasil ini mengindikasikan bahwa arsitektur pipeline dan mekanisme penyajian MJPEG yang digunakan cukup efektif untuk menangani peningkatan jumlah klien dalam skala kecil hingga menengah. Observabilitas berbasis metrik memungkinkan interpretasi hasil eksperimen secara lebih akurat, termasuk membedakan dampak peningkatan beban nyata dengan perbedaan waktu pengamatan.



\section{Sudut Pandang Mahasiswa Non-IT}

Bagian ini bertujuan melatih kemampuan membaca dan menalar perilaku sistem berdasarkan representasi visual dan konsep model, tanpa harus memahami kode sumber.

\begin{enumerate}
  \item Amati salah satu dashboard Grafana yang menampilkan metrik sistem.  
  Jelaskan perilaku sistem yang Anda amati menggunakan bahasa sehari-hari.  
  

  \item Identifikasi satu metrik yang menurut Anda paling penting bagi pengguna akhir (misalnya latensi atau jumlah data yang diproses).  
  Jelaskan mengapa metrik tersebut relevan dari sudut pandang pengguna.  
  

  \item Bandingkan dua kondisi sistem (misalnya jumlah pengguna sedikit dan banyak).  
  Apa perubahan yang terlihat, dan apa yang tidak berubah?  
  
\end{enumerate}

\section{Sudut Pandang Mahasiswa IT}

Bagian ini berfokus pada hubungan antara implementasi teknis, metrik runtime, dan representasi model sistem.

\begin{enumerate}
  \item Pilih satu metrik Prometheus yang diekspor oleh salah satu node.  
  Jelaskan bagian kode mana yang bertanggung jawab menghasilkan metrik tersebut.  
  

  \item Buat sketsa sederhana (diagram atau teks) yang memodelkan alur data dari Capturer hingga Web Server.  
  Tandai di mana metrik dihasilkan pada setiap komponen.  
  

  \item Jika Anda diminta menambahkan satu node baru (misalnya \emph{Filter}), metrik apa saja yang perlu ditambahkan agar node tersebut tetap dapat diamati dengan baik?  
	
\end{enumerate}

\section{Diskusi dan Refleksi}

Bagian ini mendorong refleksi lintas disiplin dan pengaitan langsung dengan prinsip Model-Driven Engineering.

\begin{enumerate}
  \item Diskusikan bagaimana dashboard Grafana dapat dipandang sebagai \emph{abstraksi} dari sistem yang sedang berjalan.  
  Apa keuntungan menggunakan abstraksi seperti ini dibandingkan membaca kode atau log secara langsung?

  \item Jelaskan perbedaan antara:
  \begin{itemize}
    \item sistem sebagai implementasi, dan
    \item sistem sebagai model atau representasi yang diamati melalui metrik dan visualisasi.
  \end{itemize}

  \item Refleksikan peran observabilitas dalam tahapan-tahapan rekayasa perangkat lunak (requirements gathering, analysis, desain,  pengujian, deployment, operasi).  
  Menurut Anda, pada tahap mana observabilitas memberikan nilai paling besar?
\end{enumerate}

\section{Ringkasan}

Bab ini memperkenalkan observabilitas sebagai pendekatan untuk memahami sistem terdistribusi melalui data runtime, bukan melalui inspeksi statis terhadap kode atau asumsi desain awal. Peserta mempelajari perbedaan antara monitoring dan observabilitas, serta peran metrik, log, dan trace sebagai sinyal utama untuk menalar perilaku sistem. Melalui pendekatan bottom-up, pemahaman dibangun dari apa yang benar-benar terjadi saat sistem berjalan, termasuk perubahan beban, keterlambatan, penumpukan data, dan dampaknya terhadap pengalaman pengguna.

Studi kasus yang digunakan adalah pipeline terdistribusi berbasis broker dengan komponen Capturer, Broker, Transformer, dan Web Server, yang dilengkapi layanan Prometheus tertanam pada setiap node. Prometheus eksternal mengumpulkan metrik melalui scraping dan Grafana memvisualisasikannya melalui dashboard. Eksperimen menunjukkan bahwa tiap komponen memiliki karakteristik beban kerja berbeda sesuai fungsinya, serta peningkatan jumlah pengguna MJPEG dapat diamati melalui perubahan penggunaan sumber daya tanpa perubahan signifikan pada latensi. Melalui tugas dan refleksi, peserta diajak mengaitkan observabilitas dengan prinsip-prinsip MDE, terutama gagasan bahwa dashboard dan metrik dapat dipandang sebagai model tingkat tinggi yang membantu menjembatani implementasi, eksekusi, dan perbaikan sistem secara iteratif.



	\chapter{Kontrol Runtime, Keamanan, dan Kebijakan}

\section{Tujuan Pembelajaran}
\begin{itemize}
  \item Memahami bahwa \textbf{Security dalam DevSecOps bersifat berlapis}, tidak tunggal.
  \item Mengidentifikasi berbagai tingkat kontrol keamanan dari desain hingga runtime.
  \item Menjelaskan \textbf{posisi dan batasan} keamanan yang dibahas pada bab ini.
  \item Memahami keamanan runtime sebagai mekanisme pembatas perilaku sistem yang sedang berjalan.
\end{itemize}

\section{Pengantar Security dalam DevSecOps}

DevSecOps merupakan pendekatan pengembangan sistem yang mengintegrasikan \emph{development}, \emph{security}, dan \emph{operations} ke dalam satu siklus kerja yang berkesinambungan. Berbeda dengan paradigma tradisional yang memposisikan keamanan sebagai aktivitas terpisah di akhir pengembangan, DevSecOps menempatkan security sebagai bagian inheren dari seluruh siklus hidup sistem.

Pendekatan ini berangkat dari kesadaran bahwa sistem modern bersifat kompleks, terdistribusi, dan terus berubah. Dalam konteks tersebut, keamanan tidak dapat lagi diperlakukan sebagai \emph{gate} terakhir sebelum rilis, melainkan sebagai seperangkat kontrol yang hadir sejak desain hingga sistem berjalan di lingkungan produksi.

\subsection{DevSecOps sebagai Integrasi Development, Security, dan Operations}

Dalam DevSecOps, security tidak berdiri sebagai domain eksklusif satu tim atau satu peran tertentu. Praktik keamanan terdistribusi di antara pengembang yang menulis kode, operator yang mengelola lingkungan runtime, serta kebijakan organisasi yang mengatur bagaimana sistem boleh dan tidak boleh berperilaku.

Integrasi ini mengubah cara berpikir tentang keamanan. Alih-alih bertanya \emph{“apakah sistem ini sudah diuji keamanannya?”}, DevSecOps mendorong pertanyaan \emph{“kontrol keamanan apa saja yang aktif di setiap tahap siklus sistem?”}. Dengan demikian, security menjadi sifat emergen dari banyak keputusan kecil yang konsisten, bukan hasil dari satu aktivitas tunggal.

\subsection{Mengapa Security Tidak Bisa Dipusatkan di Satu Tahap}

Tidak ada satu tahap pun dalam siklus DevOps yang mampu menanggung seluruh beban keamanan sistem. Keputusan desain yang buruk tidak dapat sepenuhnya diperbaiki oleh enkripsi di runtime. Sebaliknya, implementasi yang rapi sekalipun dapat menjadi tidak aman jika konfigurasi deployment salah atau kebijakan operasional diabaikan.

Ancaman keamanan juga bersifat lintas tahap. Kesalahan desain menciptakan permukaan serangan konseptual, kesalahan implementasi menciptakan celah teknis, sementara kesalahan operasional membuka peluang eksploitasi di dunia nyata. Oleh karena itu, memusatkan security pada satu tahap menciptakan ilusi aman yang berbahaya.

DevSecOps menekankan bahwa setiap tahap hanya mampu mengurangi sebagian risiko, dan keamanan sistem secara keseluruhan bergantung pada akumulasi kontrol lintas tahap tersebut.

\subsection{Security sebagai Kontrol, bukan Sekadar Checklist}

Salah satu kesalahpahaman umum dalam praktik keamanan adalah memandang security sebagai daftar periksa (checklist): enkripsi ada atau tidak, autentikasi ada atau tidak, firewall aktif atau tidak. Pendekatan ini cenderung bersifat statis dan simbolik.

Dalam DevSecOps, security dipahami sebagai mekanisme kontrol terhadap perilaku sistem. Kontrol ini membatasi apa yang boleh dilakukan sistem, oleh siapa, dalam kondisi apa, dan pada tingkat mana. Dengan sudut pandang ini, keamanan tidak dinilai dari keberadaan fitur, melainkan dari perubahan perilaku sistem ketika kontrol diterapkan.

Bab ini menggunakan perspektif tersebut untuk membingkai diskusi keamanan. Fokus tidak diletakkan pada kelengkapan mekanisme keamanan, melainkan pada bagaimana satu kontrol runtime dapat mengubah risiko aktual sistem yang sedang berjalan, serta bagaimana kontrol tersebut diposisikan secara realistis dalam spektrum DevSecOps yang lebih luas.


\section{Tingkatan Security dalam DevSecOps}

Keamanan dalam DevSecOps tidak hadir sebagai satu mekanisme tunggal, melainkan sebagai lapisan kontrol yang tersebar di berbagai tahap siklus hidup sistem. Setiap tahap memiliki karakteristik risiko, jenis ancaman, serta bentuk kontrol keamanan yang berbeda. Oleh karena itu, keamanan sistem bersifat kumulatif, di mana setiap tahap berkontribusi mengurangi sebagian risiko.

\subsection{Security pada Tahap Desain}

Pada tahap desain, keamanan berfokus pada cara sistem dipahami dan dibatasi secara konseptual sebelum satu baris kode pun ditulis. Keputusan pada tahap ini bersifat fundamental dan menentukan ruang lingkup risiko sistem.

\emph{Contoh}, melalui kegiatan \emph{threat modeling}, perancang sistem mengidentifikasi aset penting, alur data kritis, serta potensi ancaman yang mungkin muncul dari aktor tidak berwenang. Dari analisis ini dapat ditentukan bagian sistem mana yang memerlukan kontrol tambahan pada tahap berikutnya.

\emph{Contoh lain}, penentuan \emph{trust boundary} digunakan untuk memisahkan komponen yang dipercaya dan tidak dipercaya. Misalnya, klien eksternal dan broker publik diperlakukan sebagai bagian yang tidak dipercaya, sehingga data yang melewati batas ini harus dianggap berisiko dan memerlukan perlindungan tambahan.

\subsection{Security pada Tahap Implementasi}

Tahap implementasi menerjemahkan keputusan desain ke dalam kode dan struktur program. Keamanan pada tahap ini berfokus pada pencegahan kesalahan teknis yang dapat dieksploitasi secara langsung.

\emph{Misalnya}, setiap input yang diterima dari luar sistem divalidasi dan disanitasi sebelum diproses lebih lanjut, untuk mencegah manipulasi data atau injeksi perintah berbahaya.

\emph{Selain itu}, pengembang memilih pustaka atau framework yang telah teruji dan dipelihara dengan baik, terutama untuk fungsi kriptografi dan komunikasi jaringan, daripada mengimplementasikan mekanisme keamanan sendiri yang berpotensi lemah.

\subsection{Security pada Tahap Build dan Artifact}

Pada tahap build, fokus keamanan bergeser ke artefak yang dihasilkan dari proses pengembangan. Risiko pada tahap ini sering kali berasal dari komponen pihak ketiga yang tidak terlihat secara langsung oleh pengembang.

\emph{Contoh}, proses build dapat mencakup pemeriksaan dependensi untuk mendeteksi pustaka yang memiliki kerentanan yang telah diketahui, sehingga artefak yang dihasilkan tidak membawa risiko tersembunyi ke lingkungan produksi.

\emph{Contoh lain}, praktik \emph{image hardening} pada container dilakukan dengan menghilangkan komponen yang tidak diperlukan, sehingga artefak yang dijalankan memiliki permukaan serangan yang lebih kecil.

\subsection{Security pada Tahap Deployment dan Environment}

Tahap deployment dan pengelolaan lingkungan menentukan bagaimana sistem dihadapkan pada kondisi operasional nyata. Kesalahan konfigurasi pada tahap ini dapat membatalkan kontrol keamanan yang telah dibangun sebelumnya.

\emph{Misalnya}, setiap layanan dijalankan dengan hak akses minimum yang diperlukan, sehingga kompromi satu komponen tidak secara otomatis memberikan kendali penuh terhadap sistem lain.

\emph{Selain itu}, rahasia seperti kunci enkripsi dan kredensial tidak disimpan di dalam kode aplikasi, melainkan dikelola melalui mekanisme konfigurasi runtime atau layanan pengelola rahasia di lingkungan deployment.

\subsection{Security pada Tahap Runtime dan Eksekusi}

Tahap runtime merupakan titik di mana sistem benar-benar berinteraksi dengan data, pengguna, dan lingkungan yang tidak sepenuhnya dapat diprediksi. Kontrol keamanan pada tahap ini bertujuan membatasi perilaku sistem yang sedang berjalan.

\emph{Contoh}, enkripsi payload diterapkan pada komunikasi antar komponen sehingga pihak yang tidak berwenang masih dapat mengamati lalu lintas data, tetapi tidak dapat membaca isi informasi yang dikirimkan.

\emph{Contoh lain}, komponen tertentu dibatasi hanya pada aksi yang diizinkan selama runtime, misalnya hanya dapat membaca data tanpa kemampuan memodifikasi atau menyebarkannya lebih lanjut.

Dengan melihat tingkatan ini secara berurutan, dapat dipahami bahwa keamanan dalam DevSecOps bukanlah hasil dari satu mekanisme tunggal, melainkan dari kombinasi kontrol yang saling melengkapi di berbagai tahap siklus sistem.


\section{Posisi Bab Ini dalam Spektrum DevSecOps}

Bab ini tidak bertujuan membahas keamanan sistem secara menyeluruh. Sejumlah aspek penting dalam DevSecOps---seperti analisis kerentanan statis, pengujian penetrasi, manajemen identitas dan akses, serta keamanan jaringan tingkat rendah---tidak dibahas secara mendalam. Topik-topik tersebut tetap penting, namun berada di luar ruang lingkup pembahasan bab ini.

Pembatasan fokus dilakukan karena keamanan dalam DevSecOps bersifat luas dan berlapis. Jika seluruh lapisan dibahas sekaligus dalam satu bab, pembahasan berisiko berubah menjadi daftar fitur tanpa pemahaman konseptual yang kuat. Karena itu, bab ini memilih satu contoh kontrol yang cukup spesifik agar pembaca dapat memahami logika keamanan sebagai kontrol yang bekerja pada sistem secara nyata.

Fokus bab ini diletakkan pada keamanan runtime, yakni kontrol yang aktif ketika sistem berjalan dan berinteraksi dengan lingkungan operasional. Keamanan runtime dipilih karena berada pada titik di mana risiko aktual muncul dan dampak kontrol dapat diamati langsung melalui perubahan perilaku sistem, bukan hanya melalui klaim desain atau asumsi implementasi.

Sebagai contoh, bab ini menggunakan enkripsi payload pada sistem terdistribusi untuk menunjukkan bagaimana kebijakan keamanan dapat diterjemahkan menjadi pembatasan keterbacaan data di runtime. Enkripsi dipilih karena jelas, terlokalisasi, dan mudah diverifikasi, sekaligus menegaskan batasannya: satu kontrol tidak otomatis membuat sistem aman. Pembaca diharapkan memahami posisi kontrol runtime dalam spektrum DevSecOps yang lebih luas, serta menyadari bahwa keamanan sistem merupakan hasil kombinasi kontrol lintas tahap, bukan satu mekanisme tunggal.



\section{Pengantar Kontrol Runtime}

Kontrol keamanan dalam sistem perangkat lunak dapat dibedakan berdasarkan kapan kontrol tersebut bekerja. Sebagian kontrol bersifat statis, diterapkan sebelum sistem dijalankan, sementara kontrol lainnya bersifat dinamis dan aktif ketika sistem sedang beroperasi. Perbedaan ini penting karena jenis risiko yang dihadapi sistem berubah secara signifikan ketika sistem berpindah dari tahap perancangan dan pembangunan ke tahap eksekusi nyata.

Kontrol statis, seperti analisis desain, pemeriksaan kode, atau validasi artefak build, berfungsi mencegah kelas kesalahan tertentu sebelum sistem digunakan. Kontrol ini bekerja berdasarkan asumsi dan skenario yang dapat diprediksi. Namun, ketika sistem telah berjalan, sistem berinteraksi dengan lingkungan, data, dan aktor yang tidak sepenuhnya dapat dikendalikan atau diperkirakan. Pada titik inilah kontrol statis mencapai batasnya.

Runtime menjadi sumber risiko nyata karena hampir seluruh dampak keamanan muncul saat sistem sedang beroperasi. Akses tidak sah, kebocoran data, penyalahgunaan layanan, dan eksfiltrasi informasi terjadi pada fase ini. Risiko tersebut tidak selalu berasal dari kesalahan desain atau implementasi, tetapi sering kali muncul dari cara sistem digunakan, disalahgunakan, atau dikombinasikan dengan kondisi lingkungan tertentu.

Kontrol runtime hadir untuk membatasi perilaku sistem yang sedang berjalan. Alih-alih hanya memastikan bahwa sistem dibangun dengan benar, kontrol ini mengatur apa yang boleh dan tidak boleh dilakukan sistem dalam kondisi operasional. Dengan demikian, fokus kontrol runtime bukan pada struktur sistem, melainkan pada aksi nyata yang dihasilkan selama eksekusi.

Sebagai mekanisme pembatas, kontrol runtime tidak menghilangkan semua risiko, tetapi mengurangi dampak ketika risiko tersebut terjadi. Misalnya, kontrol dapat memastikan bahwa data yang mengalir di jaringan tidak dapat dibaca oleh pihak yang tidak berwenang, atau bahwa suatu komponen tidak dapat melakukan aksi di luar perannya meskipun berhasil dijalankan.

Pendekatan ini menempatkan keamanan sebagai properti perilaku sistem, bukan sekadar atribut desain. Efektivitas kontrol runtime dapat diamati secara empiris melalui perubahan perilaku sistem, seperti perbedaan antara sistem sebelum dan sesudah kontrol diterapkan. Karakteristik inilah yang menjadikan kontrol runtime relevan untuk dikaji melalui eksperimen dan observasi, sebagaimana akan dibahas pada bagian-bagian selanjutnya dalam bab ini.


\section{Arsitektur Sistem Studi Kasus}

Arsitektur pada Gambar~\ref{fig:broker-pubsub-runtime-encryption} menggambarkan sebuah sistem terdistribusi berbasis komunikasi \emph{publish--subscribe} dengan broker sebagai perantara pertukaran pesan antar komponen. Pola ini dipilih karena merepresentasikan sistem terbuka dengan keterikatan longgar, di mana produsen dan konsumen data tidak saling mengenal secara langsung dan bergantung pada broker untuk distribusi pesan. Dalam arsitektur ini, \emph{Capturer} berperan sebagai penghasil data mentah, \emph{Transformer} sebagai pemroses data, dan \emph{Web Server} sebagai penyaji informasi kepada pengguna akhir, sementara broker bertindak sebagai komponen netral yang hanya meneruskan pesan berdasarkan topik.

\begin{figure}[htbp]
\centering
\scalebox{0.9}{
\begin{tikzpicture}[
  font=\small\bfseries,
  node distance=9mm and 12mm,
  box/.style={
    draw,
    rounded corners=2mm,
    align=center,
    minimum width=32mm,
    minimum height=9mm,
    fill=blue!10
  },
  brokerbox/.style={box,fill=orange!15},
  webbox/.style={box,fill=green!12},
  browserbox/.style={box,fill=gray!12},
  extbox/.style={box,fill=purple!10},
  annot/.style={font=\scriptsize\itshape, text=black!70},
  arrow/.style={-Latex,thick},
  dottedarrow/.style={-Latex,thick,dotted}
]

% Core nodes
\node[box] (transformer) {Transformer\\(Node B)\\{\scriptsize Prometheus service}};
\node[annot, above=1mm of transformer] {Decrypt $\rightarrow$ Process $\rightarrow$ Encrypt};

\node[box,left=of transformer, xshift=-10mm] (capturer) {Capturer\\(Node A)\\{\scriptsize Prometheus service}};
\node[annot, above=1mm of capturer] {Encrypt};

\node[brokerbox, below=of capturer, yshift=-5mm] (broker) {Broker\\(XSUB / XPUB)\\{\scriptsize Prometheus service}};

\node[browserbox,below=of broker] (browser) {Browser};

\node[webbox,right=of browser, xshift=10mm](web) {Web Server\\(Node C)\\{\scriptsize Prometheus service}};
\node[annot, below=1mm of web] {Decrypt};

% External observability nodes
\node[extbox,right=of web] (grafana) {Grafana};
\node[extbox,above=of grafana, yshift=10mm] (prom) {Prometheus};

% Data flow arrows (encrypted payload)
\draw[arrow] (capturer) -- 
  node[above, xshift=-8mm]{PUB \texttt{raw} (encrypted)} 
  (broker);

\draw[arrow] (broker) -- 
  node[above, yshift=-5mm, xshift=8mm]{SUB \texttt{raw} (encrypted)} 
  (transformer);

\draw[arrow] (transformer.south) |- 
  node[below]{PUB \texttt{processed} (encrypted)} 
  (broker.east);

\draw[arrow] (broker) -- 
  node[right,xshift=3mm]{SUB \texttt{processed} (encrypted)} 
  (web.north);

\draw[arrow] (web.west) -- 
  node[above]{HTTP MJPEG} 
  (browser.east);

% Observability
\draw[dottedarrow] (capturer.south) -- (prom);
\draw[dottedarrow] (broker) -- (prom);
\draw[dottedarrow] (transformer.east) -- (prom);
\draw[dottedarrow] (web.east) -- (prom);
\draw[arrow] (prom) -- (grafana);

\end{tikzpicture}
}
\caption{Arsitektur sistem publish--subscribe dengan enkripsi payload sebagai kontrol keamanan runtime, di mana proses enkripsi dan dekripsi dilakukan pada node produsen dan konsumen, sementara broker tetap tidak memiliki akses terhadap isi pesan}
\label{fig:broker-pubsub-runtime-encryption}
\end{figure}

Sebagai komponen infrastruktur, broker tidak memiliki kewenangan terhadap isi pesan yang ditransmisikan. Ia tidak memahami makna payload dan tidak dibekali kemampuan untuk memodifikasi atau menginterpretasikan data. Karakteristik ini menjadikan broker sebagai titik yang aman secara fungsional namun berisiko secara observasional, karena lalu lintas pesan yang melewatinya dapat diamati oleh pihak yang memiliki akses jaringan.

Untuk membatasi risiko tersebut, kontrol keamanan diterapkan pada tingkat runtime dalam bentuk enkripsi payload. Enkripsi tidak diperlakukan sebagai fitur tambahan aplikasi, melainkan sebagai mekanisme pembatas perilaku sistem yang sedang berjalan. Pada arsitektur ini, \emph{Capturer} mengenkripsi payload sebelum mempublikasikan pesan ke broker. \emph{Transformer} mendekripsi pesan yang diterima untuk diproses, kemudian mengenkripsi kembali hasil pemrosesan sebelum dipublikasikan ulang. \emph{Web Server} mendekripsi payload yang diterima sebelum menyajikan data ke browser. Sepanjang jalur komunikasi melalui broker, payload selalu berada dalam keadaan terenkripsi.

Dengan pendekatan ini, lalu lintas pesan tetap dapat diamati, tetapi isi informasi tidak dapat dibaca tanpa kunci yang sesuai. Kontrol yang diterapkan bukan membatasi alur data, melainkan membatasi keterbacaan dan pemaknaan data oleh pihak yang tidak berwenang. Hal ini menegaskan bahwa keamanan runtime berfokus pada perilaku sistem, bukan pada struktur arsitektur atau keberadaan fitur tertentu.

Kunci enkripsi dalam arsitektur ini merepresentasikan kebijakan keamanan runtime. Kepemilikan kunci menentukan siapa yang berhak mengakses makna data, terlepas dari siapa yang mampu mengamati pesan secara teknis. Sesuai dengan prinsip DevSecOps, kunci tidak ditanamkan secara statis di dalam kode atau diberikan kepada broker, melainkan dikelola sebagai rahasia operasional yang terpisah dari logika aplikasi.

Keputusan untuk tidak memberikan kunci enkripsi kepada broker bersifat konseptual dan operasional. Secara konseptual, broker diposisikan sebagai komponen yang tidak berwenang terhadap isi data. Secara operasional, keputusan ini membatasi dampak jika broker dikompromikan, karena payload yang dilewatkan tetap tidak dapat dibaca. Dengan demikian, keamanan sistem tidak bergantung pada kepercayaan terhadap satu komponen, melainkan pada pembatasan peran dan kebijakan yang diterapkan pada runtime.

Narasi ini menunjukkan bagaimana satu kontrol keamanan runtime yang sederhana dapat menurunkan risiko akses data pada sistem terdistribusi terbuka, tanpa mengubah arsitektur dasar sistem. Pendekatan ini menegaskan bahwa keamanan dalam DevSecOps bukanlah hasil dari satu mekanisme tunggal, melainkan dari kombinasi keputusan desain, implementasi, kebijakan, dan operasi yang membatasi perilaku sistem secara nyata.


\section{Eksperimen dan Validasi Kontrol Keamanan Runtime}

Eksperimen ini menunjukkan bahwa kontrol keamanan runtime dapat dibuktikan secara empiris melalui perubahan perilaku sistem yang sedang berjalan. Fokusnya bukan pada kriptografi sebagai teori, melainkan pada bagaimana satu kebijakan sederhana---enkripsi payload---membatasi keterbacaan data pada sistem terdistribusi berbasis publish--subscribe.

Skenario eksperimen terdiri dari tiga tahap. Pertama, sistem dijalankan tanpa kontrol keamanan sehingga payload yang melewati broker dapat dibaca sebagai data mentah. Kedua, kontrol enkripsi payload ditambahkan tanpa mengubah arsitektur dasar sistem, melainkan hanya mengubah perilaku endpoint: produsen mengenkripsi sebelum publish dan konsumen mendekripsi setelah subscribe. Ketiga, dilakukan simulasi akses oleh pihak tidak berwenang (sniffer) yang hanya mampu mengamati lalu lintas pesan, untuk menunjukkan perbedaan antara data yang terlihat dan data yang terbaca.

Titik observasi utama berada pada jalur komunikasi antar komponen, khususnya pada sisi broker/jaringan. Pada kondisi tanpa enkripsi, sniffer dapat menampilkan payload dalam bentuk yang bermakna (misalnya header dan bytes JPEG dapat dikenali). Pada kondisi terenkripsi, sniffer tetap dapat melihat adanya lalu lintas dan ukuran pesan, tetapi isi payload menjadi bytes acak yang tidak dapat ditafsirkan tanpa kunci. Dengan demikian, kontrol runtime yang diterapkan menghasilkan bukti perubahan perilaku sistem yang dapat diamati.

Kontrol enkripsi diimplementasikan menggunakan pustaka \texttt{cryptography} (Fernet). Kunci enkripsi diperlakukan sebagai rahasia runtime dan diinjeksi melalui konfigurasi environment pada container, bukan ditanamkan di dalam kode. Hal ini mencerminkan prinsip DevSecOps: pemisahan antara kode, konfigurasi, dan rahasia.

\subsection*{Generasi Kunci Enkripsi sebagai Kebijakan Runtime}

Sebelum kontrol enkripsi payload diterapkan pada sistem, diperlukan kunci enkripsi yang akan berfungsi sebagai kebijakan keamanan runtime. Kunci ini tidak dihasilkan oleh aplikasi saat berjalan, melainkan dibuat terlebih dahulu sebagai bagian dari proses persiapan operasional. Dengan pendekatan ini, kebijakan keamanan dipisahkan secara tegas dari logika aplikasi dan dapat dikelola secara independen.

Pada eksperimen ini, kunci enkripsi dihasilkan menggunakan pustaka \texttt{cryptography} dengan mekanisme Fernet. Proses generasi kunci dilakukan satu kali oleh operator atau administrator sistem, kemudian hasilnya didistribusikan ke komponen yang berwenang melalui konfigurasi runtime.

\begin{lstlisting}[style=PythonStyle, caption={Generate Fernet key for payload encryption}]
from cryptography.fernet import Fernet

key = Fernet.generate_key()
print(key.decode())
\end{lstlisting}

Output dari proses ini berupa string berbasis Base64 yang merepresentasikan kunci simetris. Nilai kunci tersebut kemudian digunakan sebagai nilai variabel lingkungan \texttt{PAYLOAD\_KEY} pada layanan yang memproduksi atau mengonsumsi payload terenkripsi.

Pemisahan tahap generasi kunci dari tahap eksekusi sistem menegaskan bahwa kunci enkripsi merupakan artefak kebijakan keamanan, bukan bagian dari implementasi aplikasi. Dengan demikian, rotasi kunci, penggantian kebijakan, atau pembatasan akses dapat dilakukan tanpa perubahan kode maupun arsitektur sistem.


\subsection*{Konfigurasi Kunci pada \texttt{docker-compose.yml}}

Contoh berikut menunjukkan bahwa setiap layanan yang perlu memproduksi atau mengonsumsi payload diberi variabel \texttt{PAYLOAD\_KEY}. Broker tidak diberi kunci, karena broker diposisikan sebagai komponen netral yang tidak berwenang membaca isi data.

\begin{lstlisting}[language=bash, caption={Inject PAYLOAD\_KEY via docker-compose environment}]
services:
  transformer:
    build:
      context: .
      dockerfile: transformer/Dockerfile
    environment:
      - PAYLOAD_KEY=RmTnzrS0_DwulCzF6tj8qSOQpCmnOkRmKeezZcZA4T4=
    # ...

  capturer:
    build:
      context: .
      dockerfile: capturer/Dockerfile
    environment:
      - PAYLOAD_KEY=RmTnzrS0_DwulCzF6tj8qSOQpCmnOkRmKeezZcZA4T4=
    # ...

  web:
    build:
      context: .
      dockerfile: web/Dockerfile
    environment:
      - PAYLOAD_KEY=RmTnzrS0_DwulCzF6tj8qSOQpCmnOkRmKeezZcZA4T4=
    # ...

  broker:
    image: zmq-broker:latest
    # intentionally no PAYLOAD_KEY here
\end{lstlisting}

\subsection*{Dependensi Python pada Dockerfile}

Berikut potongan Dockerfile yang menegaskan bahwa enkripsi adalah bagian dari runtime pipeline (bersama pyzmq, metrics, dll). Contoh ini merefleksikan kebutuhan node seperti capturer/transformer yang menggunakan OpenCV, observability, dan kontrol enkripsi.

\begin{lstlisting}[language=bash, caption={Install deps including cryptography in Dockerfile}]
# ---- python deps (capturer needs OpenCV + metrics + encryption)
RUN pip install --no-cache-dir \
    pyzmq \
    opencv-python-headless \
    prometheus-client \
    psutil \
    cryptography
\end{lstlisting}

\subsection*{Fungsi Enkripsi/Dekripsi (Shared Library)}

Agar konsisten lintas layanan, praktik yang umum adalah membuat modul bersama (misalnya \texttt{shared/crypto.py}) yang membaca \texttt{PAYLOAD\_KEY} dari environment dan menyediakan fungsi \texttt{encrypt\_bytes} serta \texttt{decrypt\_bytes}.

\begin{lstlisting}[style=PythonStyle, caption={Shared crypto helpers using PAYLOAD\_KEY}]
import os
from cryptography.fernet import Fernet

_PAYLOAD_KEY = os.environ["PAYLOAD_KEY"].encode()
_cipher = Fernet(_PAYLOAD_KEY)

def encrypt_bytes(data: bytes) -> bytes:
    return _cipher.encrypt(data)

def decrypt_bytes(token: bytes) -> bytes:
    return _cipher.decrypt(token)
\end{lstlisting}

\subsection*{Integrasi pada Transformer: Decrypt--Process--Encrypt}

Pada node \emph{Transformer}, pesan masuk dari broker berada dalam bentuk terenkripsi. Transformer mendekripsi header dan JPEG, memprosesnya, lalu mengenkripsi kembali output sebelum dipublikasikan. Potongan berikut menegaskan perubahan perilaku runtime yang menjadi inti eksperimen.

\begin{lstlisting}[style=PythonStyle, caption={Transformer decrypts input and encrypts output payload}]
# receive encrypted parts from broker
enc_header_b, enc_jpeg_in = recv_parts()

# decrypt at runtime
header_b = decrypt_bytes(enc_header_b)
jpeg_in  = decrypt_bytes(enc_jpeg_in)

# process (e.g., decode -> transform -> encode)
out_header_b, jpeg_out = process_frame(header_b, jpeg_in)

# encrypt again before publishing
t_encsec = time.time()
enc_out_header = encrypt_bytes(out_header_b)
enc_jpeg_out   = encrypt_bytes(jpeg_out)

send_parts(enc_out_header, enc_jpeg_out)
\end{lstlisting}

\subsection*{Kode Uji Lokal: Pastikan Enkripsi Berfungsi dan Kunci Valid}

Pengujian berikut memastikan dua hal: payload terenkripsi tidak sama dengan payload asli, dan dekripsi hanya berhasil jika kunci benar. Ini adalah uji minimum yang membantu mahasiswa melihat konsep \emph{data terlihat tetapi tidak terbaca}.

\begin{lstlisting}[style=PythonStyle, caption={Quick test: encrypt/decrypt roundtrip}]
import os
from cryptography.fernet import Fernet, InvalidToken

key = os.environ["PAYLOAD_KEY"].encode()
cipher = Fernet(key)

plain = b"header:frame=1;ts=123456"
token = cipher.encrypt(plain)

assert token != plain
assert cipher.decrypt(token) == plain

try:
    wrong = Fernet(Fernet.generate_key())
    wrong.decrypt(token)
    raise AssertionError("Should not decrypt with wrong key")
except InvalidToken:
    pass

print("OK: encryption on, wrong key fails, correct key succeeds.")
\end{lstlisting}

\subsection*{Validasi Observasional: Sniffer sebagai Representasi Ancaman}

Validasi keamanan dilakukan melalui observasi pasif menggunakan sniffer sebagai representasi ancaman. Sniffer diposisikan sebagai pihak yang dapat mengamati lalu lintas pesan (misalnya dari sisi jaringan), tetapi tidak memiliki kunci \texttt{PAYLOAD\_KEY}. Pada kondisi tanpa enkripsi, sniffer akan memperoleh bytes yang bermakna (misalnya header atau potongan JPEG yang dapat dikenali). Setelah enkripsi diterapkan, sniffer tetap melihat payload, namun yang terlihat hanyalah token terenkripsi yang tidak dapat ditafsirkan.

Contoh berikut menunjukkan cara mencetak sebagian bytes payload untuk menegaskan perbedaan \emph{terlihat} versus \emph{terbaca} tanpa melakukan eksploitasi apa pun.

\begin{lstlisting}[style=PythonStyle, caption={Sniffer view: visible bytes, unreadable without key}]
def hexdump(b: bytes, n: int = 32) -> str:
    return b[:n].hex()

# captured payload (e.g., from subscribed wire data)
captured = token  # in practice this comes from network capture
print("Visible (first bytes):", hexdump(captured))

# attacker has no key -> cannot decrypt (demonstrated by using wrong key)
from cryptography.fernet import InvalidToken
try:
    Fernet(Fernet.generate_key()).decrypt(captured)
except InvalidToken:
    print("Unreadable: decrypt failed without the correct key.")
\end{lstlisting}

Dengan rangkaian eksperimen ini, bukti perubahan perilaku runtime menjadi jelas: aliran data tetap terjadi, broker tetap berfungsi sebagai perantara, dan observabilitas sistem tetap berjalan, tetapi keterbacaan payload bagi pihak tidak berwenang dibatasi secara nyata. Hasil ini menegaskan bahwa kontrol keamanan runtime dapat diuji dan diverifikasi secara empiris melalui observasi, serta bahwa kunci enkripsi berperan sebagai kebijakan operasional yang menghubungkan DevSecOps dengan praktik operasi sehari-hari.



\section{Sudut Pandang Mahasiswa Non-IT}

\textbf{Apa risiko yang muncul ketika sebuah sistem tidak memiliki kontrol keamanan runtime?}  
Risiko utamanya adalah data dapat diakses, dibaca, atau disalahgunakan oleh pihak yang tidak berwenang tanpa disadari oleh pengguna maupun pengelola sistem. Sistem dapat tetap berfungsi secara normal dari sudut pandang pengguna, namun informasi yang mengalir di dalamnya berada dalam kondisi rentan. Ketika kontrol keamanan tidak hadir pada saat sistem berjalan, kepercayaan terhadap sistem sebenarnya bertumpu pada asumsi, bukan pada mekanisme perlindungan yang nyata.

\textbf{Bagaimana satu kontrol keamanan sederhana dapat menurunkan risiko tersebut?}  
Penerapan kontrol seperti enkripsi payload menunjukkan bahwa tidak semua upaya keamanan harus rumit untuk memberikan dampak berarti. Dengan enkripsi, data tetap mengalir dan sistem tetap beroperasi, tetapi makna informasi tidak lagi dapat diakses oleh pihak yang tidak memiliki izin. Kontrol ini secara langsung mengubah perilaku sistem pada runtime, dari sistem yang terbuka menjadi sistem yang membatasi keterbacaan data.

\textbf{Mengapa keamanan berkaitan dengan kepercayaan dan tanggung jawab?}  
Pengguna mempercayakan data mereka kepada sistem dengan harapan bahwa pengelola bertindak secara bertanggung jawab. Ketika kontrol keamanan diterapkan, pengelola sistem menunjukkan komitmen etis dan profesional dalam melindungi informasi pengguna. Keamanan, dalam konteks ini, bukan sekadar persoalan teknis, melainkan bagian dari hubungan kepercayaan antara sistem, pengelola, dan pengguna.

\section{Sudut Pandang Mahasiswa IT}

\textbf{Di mana posisi payload encryption dalam spektrum DevSecOps?}  
Enkripsi payload berada pada lapisan keamanan runtime dan eksekusi. Kontrol ini bekerja ketika sistem sedang berjalan dan berinteraksi dengan lingkungan nyata. Dalam spektrum DevSecOps, enkripsi runtime melengkapi kontrol keamanan pada tahap desain, implementasi, build, dan deployment, namun tidak menggantikannya. Memahami posisi ini membantu menghindari anggapan bahwa satu mekanisme keamanan sudah cukup untuk menjamin keseluruhan sistem.

\textbf{Apa keterbatasan kontrol keamanan yang diterapkan pada studi kasus ini?}  
Enkripsi payload hanya mengendalikan satu dimensi risiko, yaitu kerahasiaan data selama transmisi. Kontrol ini tidak mengatur siapa yang boleh mengakses sistem, tidak mencegah penyalahgunaan layanan, dan tidak melindungi sistem dari kesalahan logika atau konfigurasi. Selain itu, efektivitas enkripsi sangat bergantung pada pengelolaan kunci yang baik. Jika kunci bocor atau disalahkelola, kontrol ini kehilangan maknanya.

\textbf{Kontrol keamanan tambahan apa yang dapat diterapkan di tingkat lain?}  
Sebagai pengembangan, kontrol tambahan dapat diterapkan pada lapisan lain, seperti autentikasi dan otorisasi untuk mengatur akses, pembatasan hak istimewa pada runtime, atau validasi konfigurasi pada tahap deployment. Usulan ini menegaskan bahwa keamanan sistem yang matang selalu dibangun dari kombinasi kontrol yang saling melengkapi di berbagai tahap.

\section{Diskusi dan Refleksi}

\textbf{Apakah satu kontrol keamanan cukup untuk menyebut suatu sistem aman?}  
Satu kontrol keamanan jarang sekali cukup. Ia dapat mengurangi satu jenis risiko, tetapi tidak mampu mencakup seluruh ancaman yang mungkin muncul. Keamanan sistem bersifat bertingkat dan bergantung pada akumulasi keputusan di berbagai lapisan. Mengandalkan satu mekanisme berisiko menciptakan rasa aman semu.

\textbf{Apa risiko jika hanya fokus pada satu lapisan keamanan?}  
Fokus pada satu lapisan dapat membuat kelemahan pada lapisan lain terabaikan. Sistem yang terenkripsi dengan baik tetap dapat disalahgunakan jika tidak memiliki kontrol akses, pemantauan, atau kebijakan operasional yang memadai. Pendekatan berlapis diperlukan untuk mencegah kegagalan pada satu titik berdampak besar terhadap keseluruhan sistem.

\textbf{Bagaimana observabilitas membantu menilai efektivitas keamanan?}  
Observabilitas memungkinkan keamanan dievaluasi berdasarkan perilaku sistem yang nyata, bukan hanya berdasarkan klaim desain atau konfigurasi. Melalui observasi, perbedaan antara data yang terlihat dan data yang terbaca dapat dibuktikan secara empiris. Dengan demikian, observabilitas menjembatani keamanan sebagai konsep teoretis dengan keamanan sebagai praktik yang dapat diuji dan ditingkatkan secara berkelanjutan.



\section{Ringkasan}

Bab ini menegaskan bahwa security dalam DevSecOps bersifat berlapis dan tidak dapat dipusatkan pada satu tahap. Keamanan dipahami sebagai mekanisme kontrol terhadap perilaku sistem, bukan sekadar checklist fitur. Melalui spektrum kontrol dari desain hingga runtime, bab ini membatasi fokus pada keamanan runtime karena pada fase inilah risiko aktual muncul dan dampak kontrol dapat diamati secara langsung. Studi kasus sistem publish--subscribe menunjukkan broker sebagai komponen netral yang tidak berwenang, sehingga risiko kebocoran muncul pada jalur komunikasi dan endpoint. Enkripsi payload dipilih sebagai contoh kontrol runtime karena sifatnya jelas, terlokalisasi, dan mampu menunjukkan perbedaan antara data yang terlihat dan data yang terbaca.

Melalui eksperimen dan validasi observasional, ditunjukkan bahwa penerapan enkripsi payload mengubah perilaku runtime tanpa mengubah arsitektur dasar sistem: produsen mengenkripsi sebelum publish, konsumen mendekripsi setelah subscribe, sementara broker tetap tidak memiliki kunci. Kunci enkripsi diposisikan sebagai kebijakan runtime yang dikelola secara operasional melalui konfigurasi environment, mencerminkan pemisahan kode, konfigurasi, dan rahasia dalam DevSecOps. Refleksi untuk mahasiswa non-IT menekankan hubungan security dengan kepercayaan dan tanggung jawab, sedangkan untuk mahasiswa IT menegaskan keterbatasan satu kontrol dan pentingnya kontrol tambahan lintas tahap. Dengan dukungan observabilitas, efektivitas kontrol keamanan dapat dinilai secara empiris dan ditingkatkan secara berkelanjutan.



	\backmatter
	\addcontentsline{toc}{chapter}{Daftar Pustaka}
	\bibliographystyle{plain}
	\bibliography{references}
	
\end{document}