\chapter{Struktur Proyek dan Perkakas Flutter}

\section{Pendahuluan}

Flutter merupakan kerangka kerja (framework) pengembangan aplikasi lintas platform yang memungkinkan pengembang membangun aplikasi mobile, web, dan desktop dari satu basis kode yang sama. Untuk dapat memanfaatkan Flutter secara efektif, seorang pengembang tidak cukup hanya memahami pembuatan antarmuka pengguna, tetapi juga harus menguasai struktur proyek, perkakas pendukung, serta alur kerja pengembangan yang disediakan oleh ekosistem Flutter.

Pertemuan pertama ini berfokus pada pemahaman dasar mengenai bagaimana sebuah proyek Flutter disusun dan bagaimana berbagai perkakas digunakan dalam proses pengembangan aplikasi. Pemahaman terhadap struktur proyek menjadi fondasi penting karena akan memengaruhi keterbacaan kode, kemudahan pemeliharaan, serta skalabilitas aplikasi di masa depan. Kesalahan pada tahap awal, seperti pengelolaan dependensi yang kurang tepat atau struktur proyek yang tidak konsisten, dapat menimbulkan masalah serius pada tahap pengembangan lanjutan.

Selain struktur proyek, Flutter juga menyediakan berbagai perkakas yang mendukung siklus hidup pengembangan aplikasi, mulai dari proses instalasi, pengujian, hingga build untuk distribusi. Melalui Flutter Command Line Interface (CLI), pengembang dapat menjalankan aplikasi, mengelola dependensi, melakukan build, serta mendiagnosis permasalahan lingkungan pengembangan secara efisien.

Pada pertemuan ini, mahasiswa diharapkan memperoleh gambaran menyeluruh mengenai ekosistem pengembangan Flutter, mencakup instalasi dan konfigurasi lingkungan pengembangan, pemahaman struktur dasar proyek, penggunaan berkas konfigurasi \texttt{pubspec.yaml}, serta konsep build dan flavor aplikasi. Materi ini menjadi dasar bagi pertemuan-pertemuan selanjutnya yang akan membahas pengembangan antarmuka, manajemen state, integrasi data, hingga optimasi performa aplikasi Flutter.


\section{Instalasi dan Lingkungan Pengembangan Flutter}

\subsection{Persyaratan Sistem}

Sebelum melakukan instalasi Flutter, penting untuk memastikan bahwa sistem yang digunakan telah memenuhi persyaratan minimum agar proses pengembangan dapat berjalan dengan lancar. Kesesuaian spesifikasi perangkat keras dan perangkat lunak akan sangat memengaruhi stabilitas lingkungan pengembangan, waktu kompilasi, serta kenyamanan dalam menjalankan emulator atau perangkat fisik.

Secara umum, Flutter dapat dijalankan pada berbagai sistem operasi utama, yaitu Windows, macOS, dan Linux. Meskipun demikian, masing-masing platform memiliki ketentuan dan ketergantungan tambahan, terutama terkait dengan pengembangan aplikasi Android dan iOS. Untuk pengembangan iOS, Flutter hanya dapat dijalankan pada sistem operasi macOS karena ketergantungan terhadap Xcode dan perangkat pengembangan iOS.

Dari sisi perangkat keras, pengembangan Flutter disarankan dilakukan pada komputer dengan prosesor modern yang mendukung arsitektur 64-bit. Kapasitas memori (RAM) yang memadai sangat penting, terutama ketika menjalankan emulator Android atau simulator iOS secara bersamaan dengan lingkungan pengembangan. Ruang penyimpanan yang cukup juga diperlukan untuk mengakomodasi Flutter SDK, Android SDK, emulator, serta dependensi proyek.

Selain perangkat keras, sistem operasi dan perangkat lunak pendukung juga harus diperhatikan. Flutter memerlukan Git untuk proses instalasi dan pembaruan SDK. Java Development Kit (JDK) dibutuhkan untuk pengembangan Android, sementara Android Studio atau IDE lain yang mendukung Flutter dapat digunakan sebagai lingkungan pengembangan utama. Untuk memastikan kompatibilitas dan kemudahan instalasi, disarankan menggunakan versi stabil terbaru dari sistem operasi dan perangkat lunak pendukung.

Dengan memastikan seluruh persyaratan sistem terpenuhi sejak awal, pengembang dapat menghindari berbagai permasalahan teknis yang sering muncul pada tahap instalasi dan konfigurasi lingkungan pengembangan. Hal ini juga akan mempermudah proses pembelajaran dan pengembangan aplikasi Flutter pada tahap-tahap selanjutnya.

\subsection{Instalasi Flutter SDK}

Flutter SDK merupakan komponen utama yang menyediakan pustaka, perkakas, dan runtime yang dibutuhkan untuk mengembangkan aplikasi Flutter. Instalasi Flutter SDK dilakukan melalui terminal atau command prompt dan dapat diterapkan pada berbagai sistem operasi seperti Windows, macOS, dan Linux.

Langkah pertama adalah mengunduh Flutter SDK resmi dari repositori Flutter. Setelah proses unduhan selesai, SDK diekstrak ke dalam direktori tertentu yang mudah diakses, misalnya direktori \texttt{home}. Contoh proses ekstraksi Flutter SDK pada sistem berbasis Unix ditunjukkan pada perintah berikut.

\begin{lstlisting}[language=bash]
$ cd ~
$ tar xf flutter_linux_3.x.x-stable.tar.xz
\end{lstlisting}

Setelah proses ekstraksi selesai, direktori Flutter perlu ditambahkan ke dalam variabel lingkungan \texttt{PATH} agar perintah \texttt{flutter} dapat dijalankan dari direktori mana pun. Penambahan \texttt{PATH} dapat dilakukan dengan mengedit berkas konfigurasi shell, seperti \texttt{.bashrc} atau \texttt{.zshrc}.

\begin{lstlisting}[language=bash]
$ export PATH="$PATH:$HOME/flutter/bin"
\end{lstlisting}

Agar perubahan konfigurasi dapat langsung digunakan, berkas konfigurasi shell perlu dimuat ulang atau sesi terminal dibuka kembali.

\begin{lstlisting}[language=bash]
$ source ~/.bashrc
\end{lstlisting}

Setelah konfigurasi \texttt{PATH} selesai, langkah selanjutnya adalah memastikan Flutter SDK telah terinstal dengan benar menggunakan perintah diagnostik \texttt{flutter doctor}. Perintah ini akan memeriksa kelengkapan lingkungan pengembangan, termasuk Flutter SDK, Android SDK, serta perkakas pendukung lainnya.

\begin{lstlisting}[language=bash]
$ flutter doctor
\end{lstlisting}

Output dari perintah \texttt{flutter doctor} akan menampilkan status setiap komponen dalam bentuk checklist. Jika terdapat komponen yang belum terinstal atau belum dikonfigurasi dengan benar, Flutter akan memberikan rekomendasi langkah perbaikan yang dapat diikuti oleh pengembang.

Flutter menyediakan beberapa kanal rilis, seperti \texttt{stable}, \texttt{beta}, dan \texttt{dev}. Untuk keperluan pembelajaran dan pengembangan aplikasi produksi, disarankan menggunakan kanal \texttt{stable}. Pengelolaan kanal rilis Flutter dapat dilakukan melalui perintah berikut.

\begin{lstlisting}[language=bash]
$ flutter channel stable
$ flutter upgrade
\end{lstlisting}

Dengan selesainya instalasi Flutter SDK dan verifikasi lingkungan pengembangan, pengembang telah menyiapkan fondasi yang kuat untuk memulai pembuatan proyek Flutter, konfigurasi IDE, serta pengembangan fitur aplikasi pada tahap-tahap selanjutnya.

\subsection{Konfigurasi Android Studio dan VS Code}

Setelah Flutter SDK berhasil diinstal, langkah berikutnya adalah menyiapkan Integrated Development Environment (IDE) yang akan digunakan untuk proses pengembangan aplikasi. Flutter mendukung berbagai IDE, namun Android Studio dan Visual Studio Code (VS Code) merupakan dua pilihan yang paling umum digunakan karena dukungan plugin yang lengkap dan integrasi yang baik dengan Flutter CLI.

\subsubsection{Konfigurasi Android Studio}

Android Studio merupakan IDE resmi untuk pengembangan aplikasi Android dan menyediakan integrasi yang erat dengan Flutter. Untuk menggunakan Flutter pada Android Studio, langkah pertama adalah memastikan bahwa Android Studio telah terinstal pada sistem. Setelah instalasi, plugin Flutter dan Dart perlu ditambahkan melalui menu pengelolaan plugin.

Setelah plugin terpasang, Android Studio akan secara otomatis mendeteksi Flutter SDK yang telah diinstal. Jika Flutter SDK belum terdeteksi, pengembang dapat mengatur lokasi SDK secara manual melalui menu pengaturan Android Studio. Selain itu, Android Studio juga menyediakan fitur pengelolaan Android SDK dan emulator yang diperlukan untuk menjalankan dan menguji aplikasi Flutter.

Pembuatan emulator Android dilakukan melalui Android Virtual Device (AVD) Manager. Emulator ini memungkinkan pengembang menjalankan dan menguji aplikasi Flutter tanpa memerlukan perangkat fisik. Disarankan untuk menggunakan image sistem yang mendukung arsitektur dan versi Android terbaru yang stabil.

Untuk memastikan seluruh komponen Android telah terkonfigurasi dengan benar, pengembang dapat kembali menjalankan perintah diagnostik Flutter.

\begin{lstlisting}[language=bash]
$ flutter doctor
\end{lstlisting}

Jika terdapat komponen Android yang belum terinstal, Flutter akan menampilkan instruksi tambahan, seperti pemasangan Android SDK atau penerimaan lisensi Android.

\begin{lstlisting}[language=bash]
$ flutter doctor --android-licenses
\end{lstlisting}

\subsubsection{Konfigurasi Visual Studio Code}

Visual Studio Code (VS Code) merupakan editor kode ringan yang banyak digunakan untuk pengembangan Flutter karena cepat dan fleksibel. Untuk menggunakan Flutter di VS Code, pengembang perlu memasang ekstensi Flutter yang secara otomatis akan menyertakan dukungan untuk bahasa Dart.

Setelah ekstensi Flutter dan Dart terpasang, VS Code akan mendeteksi Flutter SDK dan mengaktifkan berbagai fitur pendukung, seperti code completion, debugging, hot reload, dan manajemen proyek Flutter. Pengembang juga dapat memilih emulator atau perangkat fisik langsung dari status bar VS Code untuk menjalankan aplikasi.

VS Code terintegrasi langsung dengan Flutter CLI, sehingga berbagai perintah Flutter dapat dijalankan melalui terminal bawaan. Untuk memastikan konfigurasi berjalan dengan baik, pengembang dapat menjalankan proyek Flutter contoh menggunakan perintah berikut.

\begin{lstlisting}[language=bash]
$ flutter create hello_flutter
$ cd hello_flutter
$ flutter run
\end{lstlisting}

Apabila aplikasi contoh berhasil dijalankan pada emulator atau perangkat fisik, maka konfigurasi IDE dan lingkungan pengembangan Flutter dapat dianggap telah selesai dengan baik.

Dengan selesainya konfigurasi Android Studio dan VS Code, pengembang memiliki lingkungan pengembangan yang siap digunakan untuk membangun, menjalankan, dan melakukan debugging aplikasi Flutter secara efektif. Lingkungan ini akan digunakan secara konsisten pada pertemuan-pertemuan berikutnya untuk mengembangkan berbagai fitur aplikasi Flutter.

\subsection{Verifikasi Instalasi dengan Flutter Doctor}

Setelah seluruh komponen utama Flutter SDK dan lingkungan pengembangan dikonfigurasi, langkah penting berikutnya adalah melakukan verifikasi instalasi. Flutter menyediakan sebuah alat diagnostik bernama \texttt{flutter doctor} yang berfungsi untuk memeriksa kelengkapan dan kesiapan lingkungan pengembangan sebelum aplikasi Flutter dijalankan atau dibangun.

Perintah \texttt{flutter doctor} dijalankan melalui terminal atau command prompt. Perintah ini akan melakukan serangkaian pemeriksaan terhadap Flutter SDK, Dart SDK, Android SDK, perangkat pengembangan, serta berbagai perkakas pendukung lainnya.

\begin{lstlisting}[language=bash]
$ flutter doctor
\end{lstlisting}

Hasil dari perintah \texttt{flutter doctor} ditampilkan dalam bentuk daftar komponen beserta statusnya. Setiap komponen ditandai dengan simbol tertentu yang menunjukkan kondisi instalasi, seperti tanda centang untuk komponen yang telah terpasang dengan benar, tanda peringatan untuk komponen yang memerlukan konfigurasi tambahan, atau tanda silang untuk komponen yang belum tersedia.

Apabila Flutter Doctor mendeteksi bahwa beberapa komponen Android belum dikonfigurasi sepenuhnya, pengembang biasanya diminta untuk menyetujui lisensi Android SDK. Proses ini dapat dilakukan langsung melalui perintah berikut.

\begin{lstlisting}[language=bash]
$ flutter doctor --android-licenses
\end{lstlisting}

Setelah lisensi disetujui, disarankan untuk menjalankan kembali perintah \texttt{flutter doctor} guna memastikan bahwa seluruh permasalahan telah terselesaikan.

\begin{lstlisting}[language=bash]
$ flutter doctor
\end{lstlisting}

Selain pemeriksaan dasar, Flutter Doctor juga dapat digunakan untuk memverifikasi ketersediaan perangkat pengembangan, baik berupa emulator maupun perangkat fisik. Informasi ini penting untuk memastikan bahwa aplikasi Flutter dapat dijalankan dan diuji secara langsung pada lingkungan yang sesuai.

Dengan memastikan bahwa seluruh komponen yang diperiksa oleh Flutter Doctor berada dalam kondisi siap, pengembang dapat melanjutkan ke tahap pembuatan dan pengembangan aplikasi Flutter tanpa hambatan teknis yang signifikan. Verifikasi ini juga sebaiknya dilakukan secara berkala, terutama setelah pembaruan Flutter SDK atau perubahan konfigurasi lingkungan pengembangan.

\section{Struktur Dasar Proyek Flutter}

\subsection{Struktur Direktori Proyek}

Setiap proyek Flutter memiliki struktur direktori standar yang secara otomatis dihasilkan ketika sebuah proyek baru dibuat. Struktur ini dirancang untuk memisahkan kode aplikasi, konfigurasi platform, serta aset pendukung, sehingga pengembangan dan pemeliharaan aplikasi dapat dilakukan secara terorganisasi dan konsisten.

Pemahaman terhadap struktur direktori proyek Flutter merupakan hal yang sangat penting, terutama bagi pengembang pemula. Dengan memahami fungsi masing-masing direktori dan berkas, pengembang dapat mengetahui di mana kode aplikasi seharusnya ditulis, bagaimana konfigurasi platform dilakukan, serta bagaimana aset seperti gambar dan font dikelola.

Struktur dasar sebuah proyek Flutter yang baru dibuat dapat dilihat melalui perintah berikut.

\begin{lstlisting}[language=bash]
$ flutter create my_flutter_app
$ cd my_flutter_app
$ tree -L 2
\end{lstlisting}

Contoh keluaran struktur direktori proyek Flutter secara umum ditunjukkan sebagai berikut.

\begin{lstlisting}[language=bash]
my_flutter_app/
|-- android/
|-- ios/
|-- lib/
    |-- main.dart
|-- test/
|-- pubspec.yaml
|-- pubspec.lock
`-- README.md
\end{lstlisting}


Direktori \texttt{android/} berisi seluruh konfigurasi dan kode native yang diperlukan untuk membangun aplikasi Flutter pada platform Android. Di dalam direktori ini terdapat pengaturan Gradle, manifest aplikasi, serta konfigurasi build yang umumnya jarang dimodifikasi kecuali untuk kebutuhan tertentu seperti penambahan permission atau pengaturan build flavor.

Direktori \texttt{ios/} memiliki fungsi yang serupa dengan direktori Android, namun khusus untuk platform iOS. Direktori ini berisi proyek Xcode, konfigurasi signing, serta pengaturan dependensi iOS. Perubahan pada direktori ini biasanya dilakukan ketika mengatur identitas aplikasi, sertifikat, atau integrasi fitur native iOS.

Direktori \texttt{lib/} merupakan inti dari proyek Flutter. Seluruh kode aplikasi Flutter ditulis di dalam direktori ini. Berkas \texttt{main.dart} berfungsi sebagai entry point aplikasi, tempat eksekusi aplikasi Flutter dimulai. Pada proyek berskala besar, direktori \texttt{lib/} umumnya akan dikembangkan menjadi beberapa subdirektori untuk memisahkan fitur, layar, dan logika aplikasi.

Direktori \texttt{test/} digunakan untuk menyimpan seluruh berkas pengujian aplikasi Flutter, termasuk unit test, widget test, dan integration test. Pengujian yang terstruktur dengan baik pada direktori ini berperan penting dalam menjaga kualitas dan stabilitas aplikasi seiring bertambahnya kompleksitas kode.

Berkas \texttt{pubspec.yaml} merupakan berkas konfigurasi utama proyek Flutter yang digunakan untuk mendefinisikan dependensi, aset, versi SDK, serta metadata aplikasi. Sementara itu, \texttt{pubspec.lock} menyimpan versi pasti dari dependensi yang digunakan untuk menjamin konsistensi lingkungan pengembangan.

Dengan memahami struktur direktori proyek Flutter sejak awal, pengembang akan lebih mudah dalam menavigasi kode, menambahkan fitur baru, serta menjaga keteraturan proyek. Struktur ini juga menjadi dasar untuk pengembangan arsitektur aplikasi yang lebih kompleks pada pertemuan-pertemuan selanjutnya.

\subsection{Peran Berkas Utama dalam Proyek Flutter}

Selain memahami struktur direktori secara umum, pengembang Flutter juga perlu mengetahui peran dari berkas-berkas utama yang terdapat di dalam proyek. Setiap berkas memiliki fungsi spesifik yang saling terkait dalam proses pengembangan, build, dan pemeliharaan aplikasi. Pemahaman ini akan membantu pengembang menghindari kesalahan konfigurasi serta mempercepat proses pengembangan aplikasi.

Berkas \texttt{main.dart} merupakan berkas terpenting dalam proyek Flutter karena berfungsi sebagai titik masuk (entry point) aplikasi. Eksekusi aplikasi Flutter selalu dimulai dari fungsi \texttt{main()}, yang kemudian memanggil widget akar aplikasi melalui fungsi \texttt{runApp()}. Seluruh alur awal aplikasi, termasuk inisialisasi state global atau konfigurasi awal, biasanya ditempatkan pada berkas ini atau dipanggil dari berkas ini.

Berkas \texttt{pubspec.yaml} berperan sebagai berkas konfigurasi utama proyek Flutter. Di dalam berkas ini didefinisikan berbagai aspek penting, seperti versi Flutter dan Dart SDK yang digunakan, daftar dependensi eksternal, konfigurasi aset (gambar, ikon, font), serta metadata aplikasi. Setiap kali dependensi ditambahkan atau diubah, berkas ini harus diperbarui dan dependensi diunduh kembali melalui Flutter CLI.

Berkas \texttt{pubspec.lock} secara otomatis dihasilkan oleh Flutter untuk menyimpan versi pasti dari setiap dependensi yang digunakan dalam proyek. Berkas ini memastikan bahwa proyek dapat dibangun secara konsisten pada lingkungan pengembangan yang berbeda, misalnya pada komputer pengembang lain atau pada sistem Continuous Integration (CI). Umumnya berkas ini tidak diedit secara manual oleh pengembang.

Berkas \texttt{README.md} digunakan sebagai dokumentasi singkat proyek. Berkas ini biasanya berisi deskripsi aplikasi, cara menjalankan proyek, serta informasi penting lain yang berguna bagi pengembang lain yang terlibat dalam proyek. Meskipun tidak memengaruhi proses build aplikasi, dokumentasi yang baik sangat membantu dalam kerja tim dan pemeliharaan jangka panjang.

Pada platform Android dan iOS, terdapat pula berkas konfigurasi khusus yang memiliki peran penting. Pada Android, misalnya, berkas konfigurasi Gradle dan Android Manifest menentukan identitas aplikasi, permission, serta pengaturan build. Sementara itu, pada iOS, berkas konfigurasi Xcode dan \texttt{Info.plist} digunakan untuk pengaturan serupa pada platform iOS. Berkas-berkas ini umumnya dimodifikasi hanya ketika diperlukan, seperti saat menambahkan permission atau mengubah identitas aplikasi.

Dengan memahami peran masing-masing berkas utama dalam proyek Flutter, pengembang dapat bekerja secara lebih terstruktur dan efisien. Pengetahuan ini juga menjadi dasar untuk melakukan konfigurasi lanjutan, seperti pengaturan build flavor, integrasi layanan pihak ketiga, serta penerapan arsitektur aplikasi yang lebih kompleks.

\subsection{Konsep Entry Point Aplikasi}

Setiap aplikasi Flutter memiliki satu titik awal eksekusi yang disebut sebagai \textit{entry point}. Entry point merupakan bagian pertama dari kode yang dijalankan ketika aplikasi dimulai, baik pada emulator maupun perangkat fisik. Dalam Flutter, entry point aplikasi secara default didefinisikan melalui fungsi \texttt{main()} yang terdapat pada berkas \texttt{main.dart}.

Fungsi \texttt{main()} berperan sebagai penghubung antara lingkungan runtime Dart dan kerangka kerja Flutter. Ketika aplikasi dijalankan, Dart Virtual Machine akan mengeksekusi fungsi \texttt{main()} terlebih dahulu sebelum Flutter mulai membangun antarmuka pengguna. Oleh karena itu, seluruh proses inisialisasi awal aplikasi biasanya dimulai dari fungsi ini.

Di dalam fungsi \texttt{main()}, aplikasi Flutter dijalankan dengan memanggil fungsi \texttt{runApp()}. Fungsi ini menerima sebuah widget sebagai parameter dan menjadikannya sebagai widget akar (\textit{root widget}) dari seluruh widget tree aplikasi. Widget akar inilah yang kemudian menjadi dasar dalam proses rendering antarmuka pengguna.

\begin{lstlisting}[language=bash]
$ flutter run
\end{lstlisting}

Secara konseptual, alur eksekusi aplikasi Flutter dapat dipahami sebagai berikut: sistem menjalankan fungsi \texttt{main()}, kemudian \texttt{runApp()} dipanggil untuk menginisialisasi widget tree, setelah itu Flutter framework mulai melakukan proses rendering dan pengelolaan siklus hidup widget.

Selain menjalankan aplikasi, fungsi \texttt{main()} juga sering digunakan untuk melakukan konfigurasi awal sebelum antarmuka ditampilkan. Contoh konfigurasi tersebut antara lain adalah inisialisasi dependensi, pengaturan mode aplikasi (development atau production), pengaktifan logging, atau inisialisasi layanan pihak ketiga. Praktik ini memungkinkan aplikasi berada dalam kondisi siap sebelum interaksi pengguna dimulai.

Pada proyek yang lebih kompleks, fungsi \texttt{main()} dapat dikembangkan untuk mendukung berbagai skenario, seperti penggunaan build flavor atau konfigurasi lingkungan yang berbeda. Meskipun demikian, prinsip dasar entry point tetap sama, yaitu menyediakan satu titik awal yang jelas dan terkontrol untuk menjalankan aplikasi Flutter.

Dengan memahami konsep entry point aplikasi, pengembang akan memiliki gambaran yang jelas mengenai bagaimana sebuah aplikasi Flutter dimulai dan bagaimana alur eksekusi awal berlangsung. Pemahaman ini sangat penting sebagai dasar untuk mempelajari topik lanjutan, seperti manajemen state global, dependency injection, serta pengaturan arsitektur aplikasi Flutter secara keseluruhan.


\section{Konfigurasi Proyek dengan \texttt{pubspec.yaml}}
\subsection{Manajemen Dependensi}
\subsection{Pengaturan Asset dan Font}
\subsection{Versi SDK dan Constraint Dependensi}

\section{Manajemen Build dan Flavor Aplikasi}
\subsection{Konsep Build Flavor}
\subsection{Konfigurasi Flavor untuk Development}
\subsection{Konfigurasi Flavor untuk Staging}
\subsection{Konfigurasi Flavor untuk Production}

\section{Flutter CLI dan Alur Pengembangan}
\subsection{Perintah Dasar Flutter CLI}
\subsection{Menjalankan Aplikasi pada Emulator dan Perangkat Fisik}
\subsection{Proses Build Aplikasi}

\section{Ringkasan}
