\subsection{Tema Demo: Apa yang Bisa Dilakukan Kubernetes yang Tidak Bisa Dilakukan Terraform}

\textbf{Perbandingan Utama}

\begin{itemize}
    \item \textbf{Terraform} $\rightarrow$ Lifecycle Infrastruktur
    \item \textbf{Kubernetes} $\rightarrow$ Lifecycle Aplikasi
\end{itemize}

% ============================================================
\subsection{Demo 1 -- Self-Healing (Mustahil di Terraform)}

\begin{frame}[fragile]{Demo 1 -- Self-Healing}

\textbf{Langkah 1 -- Tampilkan Pod}

\begin{lstlisting}[language=bash]
kubectl get pods -n tts
\end{lstlisting}

\textbf{Langkah 2 -- Hapus Satu Pod}

\begin{lstlisting}[language=bash]
kubectl delete pod -n tts <nama-pod-api>
kubectl get pods -n tts -w
\end{lstlisting}

\textbf{Poin:}
Terraform tidak melakukan apa-apa.
Kubernetes merekonsiliasi desired state secara kontinu.

\end{frame}

% ============================================================
\subsection{Demo 2 -- Horizontal Auto Scaling (HPA)}

\begin{frame}[fragile]{Demo 2 -- Horizontal Auto Scaling}

\textbf{Tampilkan HPA}

\begin{lstlisting}[language=bash]
kubectl get hpa -n tts
\end{lstlisting}

\textbf{Buat Beban}

\begin{lstlisting}[language=bash]
python client/load_client.py --base-url http://tts.local --concurrency 50
\end{lstlisting}

atau

\begin{lstlisting}[language=bash]
hey -n 2000 -c 50 http://tts.local/api/healthz
\end{lstlisting}

\textbf{Amati Scaling}

\begin{lstlisting}[language=bash]
kubectl get hpa -n tts -w
kubectl get pods -n tts -w
\end{lstlisting}

\end{frame}

% ============================================================
\subsection{Demo 3 -- Rolling Update (Tanpa Downtime)}

\begin{frame}[fragile]{Demo 3 -- Rolling Update}

\begin{lstlisting}[language=bash]
kubectl set image -n tts deploy/api api=session05-api:new-version
kubectl rollout status -n tts deploy/api
kubectl get pods -n tts -w
\end{lstlisting}

Terraform akan destroy & recreate.
Kubernetes melakukan rolling update bertahap dengan health check.

\end{frame}

% ============================================================
\subsection{Demo 4 -- Restart Otomatis Saat Crash}

\begin{frame}[fragile]{Demo 4 -- Automatic Restart}

\begin{lstlisting}[language=bash]
kubectl exec -n tts deploy/api -- kill 1
\end{lstlisting}

Pod restart otomatis karena mekanisme controller reconciliation.

\end{frame}

% ============================================================
\subsection{Demo 5 -- Service Discovery}

\begin{frame}[fragile]{Demo 5 -- Service Discovery}

\begin{lstlisting}[language=bash]
kubectl exec -n tts deploy/web -- curl http://api:8000/healthz
\end{lstlisting}

Tidak perlu mengetahui IP.
DNS internal dan Service abstraction disediakan Kubernetes.

\end{frame}

% ============================================================
\subsection{Demo 6 -- Declarative Desired State}

\begin{frame}[fragile]{Demo 6 -- Declarative Desired State}

\textbf{Cek Replica}

\begin{lstlisting}[language=bash]
kubectl get deploy -n tts api
\end{lstlisting}

\textbf{Ubah Replica}

\begin{lstlisting}[language=bash]
kubectl scale -n tts deploy/api --replicas=10
\end{lstlisting}

Kubernetes menjaga state secara kontinu.
Terraform apply sekali lalu selesai.

\end{frame}

% ============================================================
\subsection{Demo 7 -- Switching Environment}

\begin{frame}[fragile]{Demo 7 -- Switching Environment}

\textbf{Switch ke Dev}

\begin{lstlisting}[language=bash]
kubectl delete ns tts
kubectl apply -k overlays/dev
\end{lstlisting}

\textbf{Switch ke Prod}

\begin{lstlisting}[language=bash]
kubectl delete ns tts
kubectl apply -k overlays/prod
\end{lstlisting}

\end{frame}

% ============================================================
\subsection{Demo 8 -- Koordinasi Multi-Container}

\begin{frame}{Demo 8 -- Koordinasi Multi-Container}

Komponen:
\begin{itemize}
    \item API
    \item Redis
    \item Worker
    \item MinIO
    \item Web
\end{itemize}

Semua dapat di-scale, di-restart, dan di-update secara independen.

\end{frame}